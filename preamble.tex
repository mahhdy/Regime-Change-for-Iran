% preamble.tex - Persian RTL Book Configuration
% راهنمای عملی گذار
% *** نسخه اصلاح‌شده ***

% ============================================
% DOCUMENT CLASS SETTINGS
% ============================================
\documentclass[12pt, letterpaper, openany]{book}

% ============================================
% CORE PACKAGES (ترتیب مهم است!)
% ============================================
\usepackage{fontspec}
\usepackage{graphicx}
\usepackage[table]{xcolor}        % ← اصلاح ۱: گزینه table برای \rowcolor
\usepackage{tcolorbox}
\usepackage{tikz}
\usepackage{pgfplots}
\usepackage{booktabs}
\usepackage{longtable}
\usepackage{multirow}
\usepackage{array}
\usepackage{float}
\usepackage{caption}
\usepackage{subcaption}
\usepackage{enumitem}
\usepackage{fancyhdr}
\usepackage{geometry}
\usepackage{titlesec}
\usepackage{etoolbox}
\usepackage{hyperref}

% ============================================
% TikZ LIBRARIES
% ============================================
\usetikzlibrary{
    shapes.geometric,
    arrows.meta,
    positioning,
    calc,
    backgrounds,
    fit,
    decorations.pathreplacing,
    matrix,
    shadows,
    mindmap                        % ← اصلاح ۲: اضافه شد برای concept و mindmap
}
\pgfplotsset{compat=1.18}
\tcbuselibrary{skins, breakable, theorems}

% ============================================
% PAGE GEOMETRY
% ============================================
\geometry{
    top=2.5cm,
    bottom=2.5cm,
    left=2cm,
    right=2.5cm,
    headheight=15pt                % ← اصلاح ۳: از ۱۴pt به ۱۵pt
}

% ============================================
% *** xepersian — حتماً آخرین پکیج! ***
% ============================================
\usepackage{xepersian}             % ← اصلاح ۴: منتقل شد به آخر

% ============================================
% PERSIAN FONT SETTINGS
% (باید بعد از xepersian باشد)
% ============================================
\settextfont[Scale=1.1]{XB Zar}
\setlatintextfont{Times New Roman}
\setdigitfont{XB Zar}

% Fallback if XB Zar not available - uncomment below:
% \settextfont[Scale=1.1]{Vazirmatn}
% \setdigitfont{Vazirmatn}

% ============================================
% COLOR DEFINITIONS
% ============================================
% Primary Colors
\definecolor{iranblue}{RGB}{0, 56, 147}
\definecolor{irangold}{RGB}{218, 165, 32}
\definecolor{irangreen}{RGB}{34, 139, 34}
\definecolor{iranred}{RGB}{200, 16, 46}

% Box Colors
\definecolor{kholaseblue}{RGB}{230, 240, 250}
\definecolor{kholaseborder}{RGB}{70, 130, 180}
\definecolor{naghlgold}{RGB}{255, 248, 220}
\definecolor{naghlborder}{RGB}{218, 165, 32}
\definecolor{olgoogreen}{RGB}{232, 245, 233}
\definecolor{olgooborder}{RGB}{46, 125, 50}
\definecolor{enghelabred}{RGB}{255, 235, 238}
\definecolor{enghelabborder}{RGB}{198, 40, 40}
\definecolor{empirepurple}{RGB}{243, 229, 245}
\definecolor{empireborder}{RGB}{123, 31, 162}
\definecolor{noktegray}{RGB}{245, 245, 245}
\definecolor{nokteborder}{RGB}{117, 117, 117}

% ============================================
% CUSTOM COLORED BOXES
% ============================================

% خلاصه (Summary) - Blue Box
\newtcolorbox{kholasebox}[1][]{
    enhanced,
    breakable,
    colback=kholaseblue,
    colframe=kholaseborder,
    arc=4pt,
    boxrule=1.5pt,
    left=10pt,
    right=10pt,
    top=10pt,
    bottom=10pt,
    fonttitle=\bfseries\large,
    title={خلاصه},
    attach boxed title to top right={yshift=-3mm, xshift=-5mm},
    boxed title style={colback=kholaseborder, colframe=kholaseborder},
    #1
}

% نقل قول (Quote) - Gold Box
\newtcolorbox{naghlbox}[1][]{
    enhanced,
    breakable,
    colback=naghlgold,
    colframe=naghlborder,
    arc=4pt,
    boxrule=1.5pt,
    left=15pt,
    right=15pt,
    top=10pt,
    bottom=10pt,
    borderline east={4pt}{0pt}{naghlborder},  % ← اصلاح ۵: east به‌جای west برای RTL
    #1
}

% الگو (Pattern/Lesson) - Green Box
\newtcolorbox{olgoobox}[1][]{
    enhanced,
    breakable,
    colback=olgoogreen,
    colframe=olgooborder,
    arc=4pt,
    boxrule=1.5pt,
    left=10pt,
    right=10pt,
    top=10pt,
    bottom=10pt,
    fonttitle=\bfseries\large,
    title={الگو و درس},
    attach boxed title to top right={yshift=-3mm, xshift=-5mm},
    boxed title style={colback=olgooborder, colframe=olgooborder},
    #1
}

% هشدار/انقلاب (Warning/Risk) - Red Box
\newtcolorbox{enghelabbox}[1][]{
    enhanced,
    breakable,
    colback=enghelabred,
    colframe=enghelabborder,
    arc=4pt,
    boxrule=2pt,
    left=10pt,
    right=10pt,
    top=10pt,
    bottom=10pt,
    fonttitle=\bfseries\large,
    title={هشدار},
    attach boxed title to top right={yshift=-3mm, xshift=-5mm},
    boxed title style={colback=enghelabborder, colframe=enghelabborder},
    #1
}

% نکته استراتژیک (Strategic Note) - Purple Box
\newtcolorbox{empirebox}[1][]{
    enhanced,
    breakable,
    colback=empirepurple,
    colframe=empireborder,
    arc=4pt,
    boxrule=1.5pt,
    left=10pt,
    right=10pt,
    top=10pt,
    bottom=10pt,
    fonttitle=\bfseries\large,
    title={نکته استراتژیک},
    attach boxed title to top right={yshift=-3mm, xshift=-5mm},
    boxed title style={colback=empireborder, colframe=empireborder},
    #1
}

% نکته فنی (Technical Note) - Gray Box
\newtcolorbox{noktebox}[1][]{
    enhanced,
    breakable,
    colback=noktegray,
    colframe=nokteborder,
    arc=4pt,
    boxrule=1pt,
    left=10pt,
    right=10pt,
    top=8pt,
    bottom=8pt,
    #1
}

% ============================================
% CHAPTER AND SECTION STYLING
% ============================================
\titleformat{\chapter}[display]
    {\normalfont\huge\bfseries\color{iranblue}}
    {\chaptertitlename\ \thechapter}
    {20pt}
    {\Huge}
    
\titleformat{\section}
    {\normalfont\Large\bfseries\color{iranblue}}
    {\thesection}
    {1em}
    {}

\titleformat{\subsection}
    {\normalfont\large\bfseries\color{iranblue!80}}
    {\thesubsection}
    {1em}
    {}

% ============================================
% HEADER AND FOOTER
% ============================================
\pagestyle{fancy}
\fancyhf{}
\fancyhead[RO]{\leftmark}
\fancyhead[LE]{\rightmark}
\fancyfoot[C]{\thepage}
\renewcommand{\headrulewidth}{0.4pt}
\renewcommand{\footrulewidth}{0pt}

% ============================================
% HYPERREF SETTINGS
% ============================================
\hypersetup{
    colorlinks=true,
    linkcolor=iranblue,
    filecolor=irangreen,
    urlcolor=iranblue!70,
    citecolor=irangreen,
    pdftitle={راهنمای عملی گذار},
    pdfauthor={},
    pdfsubject={استراتژی‌ها و سناریوهای رهایی از نظام استبدادی},
    pdfkeywords={ایران، گذار، دموکراسی، اپوزیسیون}
}

% ============================================
% CUSTOM COMMANDS
% ============================================
\newcommand{\keyword}[1]{\textbf{\textcolor{iranblue}{#1}}}
\newcommand{\important}[1]{\textbf{\textcolor{iranred}{#1}}}
\newcommand{\scenario}[1]{\textit{\textcolor{empireborder}{#1}}}

% ============================================
% TABLE STYLING
% ============================================
\renewcommand{\arraystretch}{1.4}
\newcolumntype{C}[1]{>{\centering\arraybackslash}p{#1}}
\newcolumntype{R}[1]{>{\raggedleft\arraybackslash}p{#1}}
\newcolumntype{L}[1]{>{\raggedright\arraybackslash}p{#1}}

% ============================================
% LIST STYLING
% ============================================
\setlist[itemize]{rightmargin=2em, itemsep=0.3em}
\setlist[enumerate]{rightmargin=2em, itemsep=0.3em}

% ============================================
% CAPTION STYLING
% ============================================
\captionsetup{
    font=small,
    labelfont=bf,
    format=hang,
    justification=centering,
    singlelinecheck=false
}

% ============================================
% CUSTOM ENVIRONMENTS FOR SCENARIOS
% ============================================
\newenvironment{scenariobox}[2][]{
    \begin{tcolorbox}[
        enhanced,
        breakable,
        colback=white,
        colframe=#1,
        arc=6pt,
        boxrule=2pt,
        left=12pt,
        right=12pt,
        top=12pt,
        bottom=12pt,
        fonttitle=\bfseries\Large,
        title={#2},
        attach boxed title to top right={yshift=-3mm, xshift=-5mm},
        boxed title style={colback=#1, colframe=#1, arc=3pt},
        shadow={2mm}{-2mm}{0mm}{black!30}
    ]
}{
    \end{tcolorbox}
}

% SWOT Table Environment
\newenvironment{swottable}{
    \begin{center}
    \begin{tabular}{|C{0.45\textwidth}|C{0.45\textwidth}|}
    \hline
    \rowcolor{olgoogreen} \textbf{نقاط قوت (S)} & \textbf{نقاط ضعف (W)} \\
    \hline
}{
    \hline
    \end{tabular}
    \end{center}
}

% ============================================
% TIKZ STYLES FOR DIAGRAMS
% ← اصلاح ۷: تغییر از \tikzstyle (منسوخ) به \tikzset
% ============================================
\tikzset{
    block/.style={rectangle, rounded corners, minimum width=3cm, minimum height=1cm, text centered, draw=iranblue, fill=kholaseblue, font=\bfseries},
    arrow/.style={thick, ->, >=stealth, color=iranblue},
    decision/.style={diamond, aspect=2, draw=irangold, fill=naghlgold, text centered, font=\bfseries},
    process/.style={rectangle, draw=irangreen, fill=olgoogreen, text centered, minimum width=2.5cm, minimum height=1cm},
    warning/.style={rectangle, rounded corners, draw=iranred, fill=enghelabred, text centered, font=\bfseries},
    % New Infographic Styles
    concept/.style={circle, draw=iranblue, fill=iranblue!10, text centered, minimum size=2cm, font=\bfseries},
    step/.style={rectangle, rounded corners=10pt, draw=iranblue!50, fill=white, text centered, minimum width=4cm, minimum height=1.5cm, thick},
    icon/.style={circle, draw=irangold, fill=irangold!20, minimum size=1cm},
    connection/.style={ultra thick, iranblue!30, -{Stealth[scale=1.5]}},
    background_rect/.style={fill=noktegray!50, rounded corners=15pt, draw=iranblue!20, dashed}
}

% Logo command for easy use
\newcommand{\booklogo}[1][0.1\textwidth]{%
    \includegraphics[width=#1]{images/cover/main-symbol.png}%
}
