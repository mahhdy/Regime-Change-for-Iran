% ch04-transition-scenarios.tex
% فصل چهارم: سناریوهای گذار

\chapter{سناریوهای گذار}
\label{ch:transition-scenarios}

\begin{center}
\begin{tikzpicture}[scale=0.9, transform shape]
    % Radar/Spider Chart representation
    \foreach \x/\label in {0/\rl{احتمال}, 72/\rl{سرعت}, 144/\rl{پایداری}, 216/\rl{هزینه انسانی}, 288/\rl{مقبولیت}} {
        \draw[gray!30] (0,0) -- (\x:4);
        \node[font=\tiny] at (\x:4.4) {\label};
    }
    \foreach \r in {1,2,3,4} \draw[gray!20] circle (\r);
    
    % Scenario 1: Non-violent (Blue)
    \draw[thick, iranblue, fill=iranblue, fill opacity=0.3] (0:3) -- (72:2.5) -- (144:3.5) -- (216:2) -- (288:4) -- cycle;
    \node[iranblue, font=\tiny\bfseries] at (144:4) {\rl{انقلاب مسالمت‌آمیز}};
    
    % Scenario 2: Armed (Red)
    \draw[thick, iranred, fill=iranred, fill opacity=0.3] (0:1.5) -- (72:1) -- (144:1) -- (216:4) -- (288:1) -- cycle;
    \node[iranred, font=\tiny\bfseries] at (216:4.5) {\rl{مقاومت مسلحانه}};
    
    \node[below=4.5cm of goal, align=center, font= \small \color{iranblue!70}] at (0,0) {\rl{ارزیابی کیفی سناریوها}};
\end{tikzpicture}
\captionof{figure}{\rl{مقایسه چندبعدی سناریوهای اصلی}}
\end{center}

\vspace{0.5cm}

\begin{naghlbox}
«آینده قابل پیش‌بینی نیست، اما می‌توان برای سناریوهای مختلف آماده شد. هنر استراتژی، انتخاب بهترین مسیر در میان ناشناخته‌هاست.»
\end{naghlbox}

\section{چارچوب تحلیل سناریوها}

\begin{noktebox}
هر سناریو بر اساس چهار معیار اصلی ارزیابی می‌شود:
\begin{enumerate}
    \item \textbf{احتمال وقوع:} چقدر محتمل است؟
    \item \textbf{هزینه انسانی:} چه تعداد قربانی؟
    \item \textbf{مدت زمان:} چقدر طول می‌کشد؟
    \item \textbf{پایداری نتیجه:} آیا دموکراسی پایدار خواهد بود؟
\end{enumerate}
\end{noktebox}

\begin{center}
\begin{tikzpicture}
    \begin{axis}[
        xlabel={\rl{هزینه}},
        ylabel={\rl{احتمال موفقیت}},
        xmin=0, xmax=10,
        ymin=0, ymax=10,
        width=12cm,
        height=10cm,
        grid=major,
        xtick={2,4,6,8},
        ytick={2,4,6,8},
        xticklabels={\rl{پایین}, \rl{متوسط}, \rl{بالا}, \rl{بسیار بالا}},
        yticklabels={\rl{پایین}, \rl{متوسط}, \rl{بالا}, \rl{بسیار بالا}},
    ]
    % Scenarios as bubbles
    \addplot[only marks, mark=*, mark size=15pt, fill=iranblue!50, opacity=0.7] coordinates {(5, 6)};
    \node[font=\tiny] at (axis cs:5,6) {\rl{انقلاب مسالمت‌آمیز}};
    
    \addplot[only marks, mark=*, mark size=12pt, fill=iranred!50, opacity=0.7] coordinates {(9, 3)};
    \node[font=\tiny] at (axis cs:9,3) {\rl{مقاومت مسلحانه}};
    
    \addplot[only marks, mark=*, mark size=10pt, fill=irangold!50, opacity=0.7] coordinates {(4, 3)};
    \node[font=\tiny] at (axis cs:4,3) {\rl{کودتا}};
    
    \addplot[only marks, mark=*, mark size=13pt, fill=olgoogreen!50, opacity=0.7] coordinates {(6, 5)};
    \node[font=\tiny] at (axis cs:6,5) {\rl{فروپاشی}};
    
    \addplot[only marks, mark=*, mark size=8pt, fill=empirepurple!50, opacity=0.7] coordinates {(2, 2)};
    \node[font=\tiny] at (axis cs:2,2) {\rl{مذاکره}};
    \end{axis}
\end{tikzpicture}
\captionof{figure}{\rl{موقعیت سناریوها در ماتریس هزینه-احتمال}}
\end{center}

\newpage

\section{سناریوی اول: انقلاب مردمی خشونت‌پرهیز}

\begin{scenariobox}{iranblue}{سناریوی ۱: انقلاب مردمی}

\subsection*{توصیف}
خیزش گسترده مردمی با تاکتیک‌های غیرخشونت‌آمیز که به سقوط نظام منجر می‌شود. الهام‌گرفته از انقلاب‌های مخملی اروپای شرقی.

\subsection*{پیش‌نیازها}
\begin{itemize}
    \item ائتلاف گسترده اپوزیسیون
    \item رهبری منسجم و قابل اعتماد
    \item یک رویداد آغازگر (Trigger Event)
    \item بسیج میلیونی در شهرهای بزرگ
    \item شکاف در نیروهای سرکوب
\end{itemize}

\subsection*{مراحل اجرا}

\begin{center}
\begin{tikzpicture}[node distance=1.8cm]
    \node[block, fill=iranblue!20] (prep) {\rl{آماده‌سازی}};
    \node[block, fill=iranblue!30, left=of prep] (trigger) {\rl{رویداد آغازگر}};
    \node[block, fill=iranblue!40, left=of trigger] (mass) {\rl{بسیج توده‌ای}};
    \node[block, fill=iranblue!50, below=of mass] (split) {\rl{شکاف در نظام}};
    \node[block, fill=iranblue!60, right=of split] (fall) {\rl{سقوط نظام}};
    \node[block, fill=olgoogreen!50, right=of fall] (trans) {\rl{حکومت انتقالی}};
    
    \draw[arrow] (prep) -- (trigger);
    \draw[arrow] (trigger) -- (mass);
    \draw[arrow] (mass) -- (split);
    \draw[arrow] (split) -- (fall);
    \draw[arrow] (fall) -- (trans);
\end{tikzpicture}
\end{center}

\end{scenariobox}

\begin{table}[H]
\centering
\caption{\rl{تحلیل هزینه-فایده سناریوی انقلاب مردمی}}
\begin{tabular}{|R{4cm}|C{3cm}|R{8cm}|}
\hline
\rowcolor{iranblue!20}
\textbf{\rl{معیار}} & \textbf{\rl{ارزیابی}} & \textbf{\rl{توضیح}} \\
\hline
\rl{احتمال} & \rl{۳۵٪} & \rl{نیازمند شرایط خاص} \\
\hline
\rl{هزینه انسانی} & \rl{متوسط تا بالا} & \rl{خطر سرکوب خونین} \\
\hline
\rl{مدت} & \rl{۶-۱۸ ماه} & \rl{پس از آغاز} \\
\hline
\rl{پایداری} & \rl{بالا} & \rl{اگر حکومت انتقالی موفق باشد} \\
\hline
\end{tabular}
\end{table}

\subsection{ریسک‌های اصلی}

\begin{enghelabbox}[title=خطرات جدی]
\begin{itemize}
    \item سرکوب خونین مشابه دی‌ماه ۱۴۰۴ یا بدتر
    \item شکست در بسیج کافی
    \item عدم شکاف در نیروهای امنیتی
    \item رقابت‌های درونی اپوزیسیون در لحظه حساس
\end{itemize}
\end{enghelabbox}

\newpage

\section{سناریوی دوم: مقاومت مسلحانه}

\begin{scenariobox}{iranred}{سناریوی ۲: مقاومت مسلحانه}

\subsection*{توصیف}
مبارزه مسلحانه علیه نظام، از چریکی تا شورش گسترده. این سناریو پرریسک‌ترین و پرهزینه‌ترین گزینه است.

\subsection*{اشکال مختلف}
\begin{enumerate}
    \item \textbf{چریکی شهری:} عملیات کوچک و پراکنده
    \item \textbf{شورش منطقه‌ای:} کنترل بخش‌هایی از کشور
    \item \textbf{جنگ داخلی:} درگیری تمام‌عیار
\end{enumerate}

\end{scenariobox}

\begin{table}[H]
\centering
\caption{\rl{تحلیل هزینه-فایده سناریوی مسلحانه}}
\begin{tabular}{|R{4cm}|C{3cm}|R{8cm}|}
\hline
\rowcolor{iranred!20}
\textbf{\rl{معیار}} & \textbf{\rl{ارزیابی}} & \textbf{\rl{توضیح}} \\
\hline
\rl{احتمال موفقیت} & \rl{۱۵٪} & \rl{بسیار پایین} \\
\hline
\rl{هزینه انسانی} & \rl{بسیار بالا} & \rl{صدها هزار تا میلیون‌ها} \\
\hline
\rl{مدت} & \rl{نامشخص} & \rl{سال‌ها تا دهه‌ها} \\
\hline
\rl{پایداری} & \rl{پایین} & \rl{ریسک دیکتاتوری نظامی} \\
\hline
\end{tabular}
\end{table}

\begin{enghelabbox}[title=هشدار جدی]
\textbf{این سناریو توصیه نمی‌شود.} تجربه سوریه و لیبی نشان می‌دهد که مسلح شدن مردم معمولاً به جنگ داخلی، ویرانی و مداخله خارجی منجر می‌شود.
\end{enghelabbox}

\section{سناریوی سوم: کودتای نظامی}

\begin{scenariobox}{irangold}{سناریوی ۳: کودتای نظامی}

\subsection*{توصیف}
گروهی از نظامیان (احتمالاً از ارتش یا حتی سپاه) علیه رهبری قیام می‌کنند.

\subsection*{پیش‌نیازها}
\begin{itemize}
    \item نارضایتی گسترده در نیروهای مسلح
    \item شکاف بین فرماندهان و رهبری
    \item فشار اقتصادی بر نظامیان
    \item حمایت یا سکوت قدرت‌های خارجی
\end{itemize}

\end{scenariobox}

\begin{noktebox}
\textbf{نمونه تاریخی:} انقلاب میخک پرتغال (۱۹۷۴) نشان می‌دهد که کودتای نظامی می‌تواند به دموکراسی منجر شود، اما این استثناست نه قاعده.
\end{noktebox}

\begin{table}[H]
\centering
\caption{\rl{تحلیل سناریوی کودتا}}
\begin{tabular}{|R{4cm}|C{3cm}|R{8cm}|}
\hline
\rowcolor{irangold!20}
\textbf{\rl{معیار}} & \textbf{\rl{ارزیابی}} & \textbf{\rl{توضیح}} \\
\hline
\rl{احتمال} & \rl{۱۰٪} & \rl{وابستگی شدید نظامیان به نظام} \\
\hline
\rl{هزینه} & \rl{متوسط} & \rl{اگر سریع باشد} \\
\hline
\rl{مدت} & \rl{سریع} & \rl{روزها تا هفته‌ها} \\
\hline
\rl{پایداری} & \rl{متوسط} & \rl{ریسک دیکتاتوری نظامی} \\
\hline
\end{tabular}
\end{table}

\section{سناریوی چهارم: فروپاشی از درون}

\begin{scenariobox}{olgoogreen}{سناریوی ۴: فروپاشی}

\subsection*{توصیف}
نظام به دلیل تضادها و بحران‌های درونی از هم می‌پاشد. مشابه فروپاشی شوروی.

\subsection*{عوامل محرک}
\begin{itemize}
    \item بحران جانشینی رهبری
    \item ورشکستگی اقتصادی
    \item شورش‌های توده‌ای پی‌درپی
    \item تشدید تنش‌های جناحی
    \item از دست دادن متحدان منطقه‌ای
\end{itemize}

\end{scenariobox}

\begin{center}
\begin{tikzpicture}
    % Domino effect visualization
    \foreach \i/\label in {0/\rl{بحران اقتصادی}, 1/\rl{انزوای جهانی}, 2/\rl{شورش داخلی}, 3/\rl{شکاف جناحی}, 4/\rl{بحران جانشینی}} {
        \draw[fill=iranblue!\the\numexpr30+\i*15\relax, rotate=\i*5] (\i*2, 0) rectangle (\i*2+0.8, 2.5);
        \node[font=\tiny, rotate=90] at (\i*2+0.4, 1.25) {\label};
    }
    \draw[thick, ->] (10, 1.25) -- (11, 1.25) node[right] {\rl{سقوط}};
\end{tikzpicture}
\captionof{figure}{\rl{اثر دومینویی فروپاشی}}
\end{center}

\begin{table}[H]
\centering
\begin{tabular}{|R{4cm}|C{3cm}|R{8cm}|}
\hline
\rowcolor{olgoogreen!30}
\textbf{\rl{معیار}} & \textbf{\rl{ارزیابی}} & \textbf{\rl{توضیح}} \\
\hline
\rl{احتمال} & \rl{۲۵٪} & \rl{بستگی به تحولات داخلی} \\
\hline
\rl{هزینه} & \rl{بالا} & \rl{خطر هرج‌ومرج} \\
\hline
\rl{مدت} & \rl{متغیر} & \rl{ماه‌ها تا سال‌ها} \\
\hline
\rl{پایداری} & \rl{متوسط} & \rl{نیازمند آمادگی اپوزیسیون} \\
\hline
\end{tabular}
\end{table}

\section{سناریوی پنجم: گذار مذاکره‌ای}

\begin{scenariobox}{empireborder}{سناریوی ۵: گذار مذاکره‌ای}

\subsection*{توصیف}
نظام تحت فشار به مذاکره با اپوزیسیون تن می‌دهد و انتقال قدرت تدریجی صورت می‌گیرد.

\subsection*{پیش‌نیازها}
\begin{itemize}
    \item تغییر در محاسبات هزینه-فایده نظام
    \item تضمین‌های امنیتی برای مقامات
    \item فشار بین‌المللی هماهنگ
    \item اپوزیسیون متحد و مشروع به عنوان طرف مذاکره
\end{itemize}

\end{scenariobox}

\begin{noktebox}
\textbf{نمونه موفق:} آفریقای جنوبی، جایی که نظام آپارتاید پس از سال‌ها مذاکره قدرت را واگذار کرد. اما شرایط ایران بسیار متفاوت است.
\end{noktebox}

\begin{table}[H]
\centering
\begin{tabular}{|R{4cm}|C{3cm}|R{8cm}|}
\hline
\rowcolor{empirepurple!20}
\textbf{\rl{معیار}} & \textbf{\rl{ارزیابی}} & \textbf{\rl{توضیح}} \\
\hline
\rl{احتمال} & \rl{۱۵٪} & \rl{نظام اراده‌ای برای مذاکره ندارد} \\
\hline
\rl{هزینه} & \rl{پایین} & \rl{کم‌خشونت‌ترین گزینه} \\
\hline
\rl{مدت} & \rl{طولانی} & \rl{سال‌ها} \\
\hline
\rl{پایداری} & \rl{بالا} & \rl{اگر توافق جامع باشد} \\
\hline
\end{tabular}
\end{table}

\section{سناریوی ترکیبی: واقع‌بینانه‌ترین مسیر}

\begin{empirebox}[title=تحلیل کلیدی]
در عمل، گذار احتمالاً ترکیبی از چند سناریو خواهد بود. محتمل‌ترین مسیر:

\textbf{فشار مردمی + فروپاشی تدریجی + احتمالاً مذاکره در مراحل پایانی}
\end{empirebox}

\begin{center}
\begin{tikzpicture}[node distance=2cm, scale=0.9, transform shape]
    % Multiple paths converging
    \node[block, fill=iranblue!30] (pressure) at (0, 3) {\rl{فشار مردمی}};
    \node[block, fill=olgoogreen!30] (fracture) at (0, 0) {\rl{شکاف درونی}};
    \node[block, fill=irangold!30] (intl) at (0, -3) {\rl{فشار جهانی}};
    
    \node[block, fill=iranred!20] (crisis) at (6, 0) {\rl{بحران سیستمی}};
    
    \node[decision] (dec) at (10, 0) {\rl{واکنش؟}};
    
    \node[block, fill=empirepurple!30] (negotiate) at (13, 2) {\rl{مذاکره خروج}};
    \node[block, fill=iranred!30] (collapse) at (13, -2) {\rl{فروپاشی}};
    
    \node[block, fill=olgoogreen!50] (trans) at (16, 0) {\rl{حکومت انتقالی}};
    
    \draw[arrow] (pressure) -- (crisis);
    \draw[arrow] (fracture) -- (crisis);
    \draw[arrow] (intl) -- (crisis);
    \draw[arrow] (crisis) -- (dec);
    \draw[arrow] (dec) -- node[above, font=\tiny] {\rl{عقلانی}} (negotiate);
    \draw[arrow] (dec) -- node[below, font=\tiny] {\rl{لجاجت}} (collapse);
    \draw[arrow] (negotiate) -- (trans);
    \draw[arrow] (collapse) -- (trans);
\end{tikzpicture}
\captionof{figure}{\rl{مسیر ترکیبی احتمالی گذار}}
\end{center}

\section{ماتریس مقایسه‌ای سناریوها}

\begin{table}[H]
\centering
\caption{\rl{جدول مقایسه‌ای کامل سناریوها}}
\resizebox{\textwidth}{!}{
\begin{tabular}{|R{4cm}|C{2.5cm}|C{2.5cm}|C{2.5cm}|C{2.5cm}|C{3cm}|}
\hline
\rowcolor{iranblue!30}
\textbf{\rl{سناریو}} & \textbf{\rl{احتمال}} & \textbf{\rl{هزینه}} & \textbf{\rl{سرعت}} & \textbf{\rl{پایداری}} & \textbf{\rl{توصیه}} \\
\hline
\rl{انقلاب خشونت‌پرهیز} & \rl{۳۵٪} & \rl{متوسط} & \rl{۶-۱۸ ماه} & \rl{بالا} & \cellcolor{olgoogreen!30}\rl{اولویت} \\
\hline
\rl{مقاومت مسلحانه} & \rl{۱۵٪} & \rl{بسیار بالا} & \rl{نامشخص} & \rl{پایین} & \cellcolor{iranred!30}\rl{اجتناب} \\
\hline
\rl{کودتا} & \rl{۱۰٪} & \rl{متوسط} & \rl{سریع} & \rl{متوسط} & \cellcolor{irangold!30}\rl{غیرقابل برنامه‌ریزی} \\
\hline
\rl{فروپاشی} & \rl{۲۵٪} & \rl{بالا} & \rl{متغیر} & \rl{متوسط} & \cellcolor{iranblue!20}\rl{آمادگی} \\
\hline
\rl{مذاکره} & \rl{۱۵٪} & \rl{پایین} & \rl{طولانی} & \rl{بالا} & \cellcolor{empirepurple!20}\rl{در صورت امکان} \\
\hline
\textbf{\rl{ترکیبی}} & - & \rl{متوسط} & - & \rl{بالا} & \cellcolor{olgoogreen!50}\textbf{\rl{واقع‌بینانه}} \\
\hline
\end{tabular}
}
\end{table}

\begin{kholasebox}[title=خلاصه فصل]
\begin{itemize}
    \item پنج سناریوی اصلی برای گذار قابل تصور است
    \item انقلاب خشونت‌پرهیز بیشترین احتمال موفقیت را دارد
    \item مقاومت مسلحانه به شدت توصیه نمی‌شود
    \item سناریوی ترکیبی واقع‌بینانه‌ترین مسیر است
    \item آمادگی برای چندین سناریو ضروری است
\end{itemize}
\end{kholasebox}
