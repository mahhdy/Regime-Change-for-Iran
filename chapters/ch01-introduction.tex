% ch01-introduction.tex
% فصل اول: مقدمه

\chapter{مقدمه}
\label{ch:introduction}

\begin{center}
\begin{tikzpicture}[scale=0.9, transform shape]
    % Road
    \draw[thick, iranblue!30, fill=noktegray] (-5, -2) -- (5, -2) -- (1, 3) -- (-1, 3) -- cycle;
    \draw[dashed, iranblue!50] (0, -2) -- (0, 3);
    
    % Signpost
    \draw[ultra thick, brown] (0, 0.5) -- (0, -0.5);
    \node[rectangle, draw=iranblue, fill=white, rounded corners, minimum width=3cm, rotate=15] at (2, 1.5) {\rl{دموکراسی و صلح}};
    \node[rectangle, draw=iranred, fill=white, rounded corners, minimum width=3cm, rotate=-15] at (-2, 1.5) {\rl{فروپاشی و جنگ}};
    
    % Label
    \node[concept, scale=0.7] at (0, -3) {\rl{ایران در دو راهی}};
    
    % Arrows showing choice
    \draw[connection, iranblue] (0, -1) .. controls (1, 0) .. (2, 1);
    \draw[connection, iranred] (0, -1) .. controls (-1, 0) .. (-2, 1);
\end{tikzpicture}
\captionof{figure}{\rl{ایران در دوراهی سرنوشت‌ساز}}
\end{center}

\vspace{1cm}

\begin{naghlbox}
«تاریخ به ما می‌آموزد که هیچ نظام استبدادی ابدی نیست. سوال این نیست که آیا تغییر خواهد آمد، بلکه سوال این است که چگونه و با چه هزینه‌ای.»
\end{naghlbox}

\section{چرا این راهنما؟}

ایران در مقطعی تاریخی قرار گرفته که نیازمند تحلیل دقیق و برنامه‌ریزی استراتژیک است. این راهنما با هدف پر کردن خلأ موجود در ادبیات فارسی درباره \keyword{استراتژی‌های گذار} تهیه شده است.

\subsection{ضرورت و فوریت}

\begin{itemize}
    \item \textbf{بحران مشروعیت نظام:} جمهوری اسلامی با عمیق‌ترین بحران مشروعیت تاریخ خود مواجه است
    \item \textbf{تغییر نسلی:} نسل جوان فاقد هرگونه وابستگی به نظام است
    \item \textbf{فشار بین‌المللی:} محیط بین‌المللی نسبت به گذشته مساعدتر است
    \item \textbf{خلأ استراتژیک:} اپوزیسیون فاقد نقشه راه منسجم است
\end{itemize}

\begin{olgoobox}[title=درس تاریخی]
تجربه انقلاب ۱۳۵۷ نشان می‌دهد که سرنگونی یک نظام بدون برنامه‌ریزی برای «روز بعد» می‌تواند به استبداد جدیدی منجر شود. این راهنما تلاش دارد از تکرار آن اشتباه جلوگیری کند.
\end{olgoobox}

\section{رویداد دی‌ماه ۱۴۰۴: نقطه عطف}

\begin{center}
\begin{tikzpicture}
    % Timeline
    \draw[thick, iranblue] (0,0) -- (14,0);
    
    % Events
    \foreach \x/\year/\event in {
        0/۱۳۵۷/\rl{انقلاب},
        2/۱۳۷۸/\rl{۱۸ تیر},
        4/۱۳۸۸/\rl{جنبش سبز},
        6.5/۱۳۹۶/\rl{دی‌ماه},
        9/۱۴۰۱/\rl{مهسا},
        12/۱۴۰۴/\rl{دی‌ماه}
    }{
        \draw[thick, iranblue] (\x, -0.2) -- (\x, 0.2);
        \node[above, font=\small\bfseries] at (\x, 0.3) {\year};
        \node[below, font=\footnotesize, text width=1.5cm, align=center] at (\x, -0.4) {\event};
    }
    
    % Highlight last event
    \draw[fill=iranred, iranred] (12,0) circle (0.25);
\end{tikzpicture}
\captionof{figure}{\rl{خط زمانی اعتراضات مهم ایران}}
\end{center}

\subsection{چه اتفاقی افتاد؟}

رویدادهای دی‌ماه ۱۴۰۴ از نظر شدت سرکوب و گستردگی اعتراضات، نقطه عطفی در تاریخ جمهوری اسلامی محسوب می‌شود:

\begin{enghelabbox}[title=آمار تکان‌دهنده]
\begin{itemize}
    \item بیش از ۱۵۰۰ کشته (طبق گزارش‌های مستقل)
    \item هزاران بازداشتی
    \item قطع کامل اینترنت به مدت یک هفته
    \item استفاده از تیراندازی مستقیم به معترضان
\end{itemize}
\end{enghelabbox}

\subsection{پیامدها}

\begin{enumerate}
    \item \textbf{قطع رابطه نهایی ملت و حکومت:} حتی معتقدان به اصلاحات درون‌نظام دیگر امیدی ندارند
    \item \textbf{افزایش فشار بین‌المللی:} کشورهای غربی موضع سخت‌تری گرفتند
    \item \textbf{بحران اقتصادی عمیق‌تر:} تحریم‌های جدید و فرار سرمایه
    \item \textbf{شکاف در درون نظام:} تنش بین جناح‌ها تشدید شد
\end{enumerate}

\section{مخاطبان این راهنما}

این راهنما برای طیف گسترده‌ای از مخاطبان نوشته شده است:

\begin{center}
\begin{tikzpicture}[
    mindmap,
    concept color=iranblue!50,
    every node/.style={concept, circular drop shadow, minimum size=2cm},
    level 1 concept/.append style={sibling angle=72, font=\small},
    level 2/.append style={minimum size=1.2cm, sibling angle=45, font=\tiny}
]
    \node[concept, scale=0.8] {\rl{مخاطبان}}
        child[concept color=irangreen!50] { node {\rl{فعالان سیاسی}} }
        child[concept color=irangold!50] { node {\rl{رهبران اپوزیسیون}} }
        child[concept color=empireborder!30] { node {\rl{پژوهشگران}} }
        child[concept color=iranred!30] { node {\rl{دیپلمات‌ها}} }
        child[concept color=nokteborder!30] { node {\rl{شهروندان علاقه‌مند}} };
\end{tikzpicture}
\end{center}

\section{روش‌شناسی و رویکرد}

\subsection{منابع مورد استفاده}

\begin{itemize}
    \item \textbf{نظریه‌های آکادمیک:} اسکاچپول، تیلی، شارپ و نظریه‌های گذار دموکراتیک
    \item \textbf{مطالعات موردی:} تحلیل ۱۲ نمونه گذار در تاریخ معاصر
    \item \textbf{داده‌های میدانی:} گزارش‌های سازمان‌های حقوق بشری
    \item \textbf{مصاحبه‌ها:} گفتگو با فعالان و کارشناسان (محرمانه)
\end{itemize}

\subsection{رویکرد تحلیلی}

\begin{noktebox}
این راهنما از رویکرد \textbf{واقع‌گرایانه} پیروی می‌کند: نه خوش‌بینی بی‌اساس و نه بدبینی فلج‌کننده. هدف ارائه تحلیلی است که بتواند مبنای تصمیم‌گیری عملی قرار گیرد.
\end{noktebox}

\section{محدودیت‌ها و هشدارها}

\begin{enghelabbox}[title=هشدارهای مهم]
\begin{enumerate}
    \item \textbf{عدم قطعیت:} آینده‌نگاری سیاسی همواره با عدم قطعیت همراه است
    \item \textbf{پویایی وضعیت:} شرایط به سرعت تغییر می‌کند؛ این تحلیل بر اساس داده‌های بهمن ۱۴۰۴ است
    \item \textbf{ریسک امنیتی:} استفاده از این راهنما در داخل ایران می‌تواند خطرناک باشد
    \item \textbf{عدم مسئولیت:} نویسندگان مسئولیتی در قبال استفاده عملی از این راهنما ندارند
\end{enumerate}
\end{enghelabbox}

\section{نحوه استفاده از این کتاب}

\begin{table}[H]
\centering
\caption{\rl{نقشه راهنمای مطالعه}}
\begin{tabular}{|C{3cm}|R{9cm}|C{4cm}|}
\hline
\rowcolor{iranblue!20}
\textbf{\rl{سطح}} & \textbf{\rl{توضیح}} & \textbf{\rl{فصول پیشنهادی}} \\
\hline
\rl{سریع} & \rl{خلاصه وضعیت و توصیه‌ها} & \rl{فصل ۰} \\
\hline
\rl{متوسط} & \rl{درک کلی از وضعیت و سناریوها} & \rl{فصول ۱-۴} \\
\hline
\rl{عمیق} & \rl{تحلیل جامع و آمادگی اجرایی} & \rl{همه فصول} \\
\hline
\rl{تخصصی} & \rl{پژوهش و ارجاع آکادمیک} & \rl{پیوست‌ها} \\
\hline
\end{tabular}
\end{table}

\begin{kholasebox}[title=خلاصه فصل]
\begin{itemize}
    \item این راهنما ابزاری برای تفکر استراتژیک درباره گذار سیاسی در ایران است
    \item رویدادهای اخیر نقطه عطفی در تاریخ جمهوری اسلامی هستند
    \item هدف ارائه تحلیل واقع‌بینانه، نه خوش‌بینانه یا بدبینانه است
    \item مخاطبان متنوعی می‌توانند از این راهنما بهره ببرند
    \item امنیت شخصی در استفاده از این کتاب را جدی بگیرید
\end{itemize}
\end{kholasebox}
