% ch08-risk-analysis.tex
% فصل هشتم: تحلیل ریسک

\chapter{تحلیل ریسک}
\label{ch:risk-analysis}

\begin{center}
\begin{tikzpicture}[scale=0.9, transform shape]
    % Heatmap background
    \fill[iranred!20] (0,3) rectangle (3,6); % High impact, high probability
    \fill[irangold!20] (0,0) rectangle (3,3); % Low impact, high probability
    \fill[irangold!20] (3,3) rectangle (6,6); % High impact, low probability
    \fill[irangreen!20] (3,0) rectangle (6,3); % Low impact, low probability
    
    \draw[thick, <->] (0, 0) -- (6.5, 0) node[right] {\rl{احتمال}};
    \draw[thick, <->] (0, 0) -- (0, 6.5) node[above] {\rl{تأثیر}};
    
    % Points
    \node[circle, fill=iranred, inner sep=2pt, label=right:\tiny{\rl{سرکوب}}] at (2, 5.5) {};
    \node[circle, fill=iranred!70, inner sep=2pt, label=right:\tiny{\rl{جنگ داخلی}}] at (1, 5) {};
    \node[circle, fill=irangold, inner sep=2pt, label=right:\tiny{\rl{تجزیه}}] at (0.5, 5.8) {};
    \node[circle, fill=irangreen, inner sep=2pt, label=right:\tiny{\rl{اقتصاد}}] at (4, 2) {};
    
    \node[below=3.5cm of goal, align=center, font= \small \color{iranred!70}] at (3,0) {\rl{نقشه حرارتی ریسک‌های استراتژیک}};
\end{tikzpicture}
\captionof{figure}{\rl{ارزیابی توزیع ریسک‌های گذار}}
\end{center}

\begin{naghlbox}
«مدیریت ریسک یعنی آمادگی برای بدترین و تلاش برای بهترین.»
\end{naghlbox}

\section{ماتریس ریسک}

\begin{center}
\begin{tikzpicture}
    % Grid
    \draw[step=2cm, gray, thin] (0,0) grid (10,10);
    
    % Axes
    \draw[thick, ->] (0,0) -- (10.5,0) node[right] {\rl{احتمال}};
    \draw[thick, ->] (0,0) -- (0,10.5) node[above] {\rl{تأثیر}};
    
    % Labels
    \foreach \x/\label in {1/\rl{کم}, 3/\rl{متوسط}, 5/\rl{بالا}, 7/\rl{بسیار بالا}, 9/\rl{حداکثر}} {
        \node[font=\tiny] at (\x, -0.3) {\label};
        \node[font=\tiny] at (-0.5, \x) {\label};
    }
    
    % Color zones
    \fill[green!20] (0,0) rectangle (4,4);
    \fill[yellow!30] (4,0) rectangle (10,4);
    \fill[yellow!30] (0,4) rectangle (4,10);
    \fill[orange!30] (4,4) rectangle (8,8);
    \fill[red!30] (8,4) rectangle (10,10);
    \fill[red!30] (4,8) rectangle (10,10);
    
    % Risks (R1-R5)
    \node[draw, circle, fill=white, font=\tiny] at (7, 9) {R1};
    \node[draw, circle, fill=white, font=\tiny] at (5, 7) {R2};
    \node[draw, circle, fill=white, font=\tiny] at (3, 8) {R3};
    \node[draw, circle, fill=white, font=\tiny] at (6, 5) {R4};
    \node[draw, circle, fill=white, font=\tiny] at (4, 3) {R5};
\end{tikzpicture}
\captionof{figure}{\rl{ماتریس ریسک استراتژیک}}
\end{center}

\begin{table}[H]
\centering
\begin{tabular}{|C{1.5cm}|R{6.5cm}|C{3cm}|C{3cm}|}
\hline
\rowcolor{iranred!20}
\textbf{\rl{کد}} & \textbf{\rl{ریسک}} & \textbf{\rl{احتمال}} & \textbf{\rl{تأثیر}} \\
\hline
\rl{R1} & \rl{سرکوب خونین‌تر} & \rl{بالا} & \rl{حداکثر} \\
\hline
\rl{R2} & \rl{جنگ داخلی} & \rl{متوسط} & \rl{بسیار بالا} \\
\hline
\rl{R3} & \rl{تجزیه کشور} & \rl{کم} & \rl{حداکثر} \\
\hline
\rl{R4} & \rl{دیکتاتوری جدید} & \rl{متوسط} & \rl{بالا} \\
\hline
\rl{R5} & \rl{شکست اقتصادی} & \rl{متوسط} & \rl{متوسط} \\
\hline
\end{tabular}
\caption{\rl{فهرست ریسک‌های اصلی}}
\end{table}

\section{سناریوهای بدترین حالت}

\subsection{سناریو ۱: سرکوب خونین}

\begin{enghelabbox}
نظام با استفاده از تمام ظرفیت سرکوب، اعتراضات را سرکوب می‌کند.
\begin{itemize}
    \item \textbf{تلفات:} هزاران تا ده‌ها هزار
    \item \textbf{پیامد:} سکوت موقت، خشم انباشته
    \item \textbf{کاهش:} تاکتیک‌های پراکنده، اجتناب از تجمعات متمرکز
\end{itemize}
\end{enghelabbox}

\subsection{سناریو ۲: جنگ داخلی}

\begin{enghelabbox}
درگیری مسلحانه بین گروه‌های مختلف.
\begin{itemize}
    \item \textbf{تلفات:} صدها هزار
    \item \textbf{پیامد:} ویرانی، مداخله خارجی
    \item \textbf{کاهش:} اجتناب از مسلح شدن، وحدت اپوزیسیون
\end{itemize}
\end{enghelabbox}

\subsection{سناریو ۳: تجزیه}

\begin{enghelabbox}
جدایی مناطق قومی.
\begin{itemize}
    \item \textbf{احتمال:} پایین اما موجود
    \item \textbf{کاهش:} تضمین حقوق اقلیت‌ها، فدرالیسم
\end{itemize}
\end{enghelabbox}

\section{استراتژی‌های کاهش ریسک}

\begin{table}[H]
\centering
\begin{tabular}{|R{4cm}|R{11cm}|}
\hline
\rowcolor{olgoogreen!30}
\textbf{\rl{ریسک}} & \textbf{\rl{استراتژی کاهش}} \\
\hline
\rl{سرکوب} & \rl{تاکتیک‌های غیرمتمرکز، حمایت بین‌المللی} \\
\hline
\rl{جنگ داخلی} & \rl{وحدت اپوزیسیون، خلع سلاح} \\
\hline
\rl{تجزیه} & \rl{فدرالیسم، حقوق برابر} \\
\hline
\rl{دیکتاتوری} & \rl{نظارت بین‌المللی، نهادهای قوی} \\
\hline
\end{tabular}
\caption{\rl{استراتژی‌های کاهش ریسک}}
\end{table}

\begin{kholasebox}[title=خلاصه فصل]
\begin{itemize}
    \item ریسک‌های جدی وجود دارند و باید مدیریت شوند
    \item سرکوب خونین محتمل‌ترین ریسک است
    \item جنگ داخلی و تجزیه باید به شدت از آنها اجتناب شود
    \item استراتژی‌های کاهش از قبل باید آماده باشند
\end{itemize}
\end{kholasebox}
