% ch03-opposition-mapping.tex
% فصل سوم: طیف‌شناسی اپوزیسیون

\chapter{طیف‌شناسی اپوزیسیون}
\label{ch:opposition-mapping}

\begin{center}
\begin{tikzpicture}[scale=0.9, transform shape]
    % Axes
    \draw[thick, <->] (-5, 0) -- (5, 0) node[right] {راست};
    \draw[thick, <->] (0, -4) -- (0, 4) node[above] {سکولار};
    \node at (-5.5, 0) {چپ};
    \node at (0, -4.5) {مذهبی};
    
    % Spectrum Circles
    \fill[iranred!20, opacity=0.6] (-3, 2) circle (1.2cm);
    \node[font=\footnotesize, align=center] at (-3, 2) {چپ\\سکولار};
    
    \fill[iranblue!20, opacity=0.6] (3, 2) circle (1.5cm);
    \node[font=\footnotesize, align=center] at (3, 2) {مشروطه و\\راست دموکرات};
    
    \fill[irangold!20, opacity=0.6] (0, 2) circle (1cm);
    \node[font=\footnotesize, align=center] at (0, 2) {جمهوری‌خواه\\میانه};
    
    \fill[irangreen!20, opacity=0.6] (-1, -2) circle (1.2cm);
    \node[font=\footnotesize, align=center] at (-1, -2) {ملی-مذهبی\\تحول‌خواه};
    
    \fill[empirepurple!20, opacity=0.6] (3, -1) circle (1cm);
    \node[font=\footnotesize, align=center] at (3, -1) {جنبش‌های\\قومی};
    
    \draw[dashed, iranblue!30] (-5,-4) rectangle (5,4);
    
    \node[below=4.5cm of goal, align=center, font= \small \color{iranblue!70}] at (0,0) {نقشه جریانات بر اساس ایدئولوژی و لائیسیته};
\end{tikzpicture}
\captionof{figure}{طیف سیاسی نیروهای اپوزیسیون}
\end{center}

\vspace{0.5cm}

\begin{naghlbox}
«اتحاد در برابر دشمن مشترک، نه به معنای یکسان‌سازی، بلکه به معنای هم‌افزایی تفاوت‌هاست.»
\end{naghlbox}

\section{نقشه جریانات اپوزیسیون}

\begin{center}
\begin{tikzpicture}[scale=0.9, transform shape]
    % Political spectrum
    \draw[thick, <->] (-7, 0) -- (7, 0);
    \draw[thick, <->] (0, -5) -- (0, 5);
    
    % Labels
    \node[font=\small] at (-6.5, -0.5) {چپ};
    \node[font=\small] at (6.5, -0.5) {راست};
    \node[font=\small, rotate=90] at (-0.5, 4.5) {سکولار};
    \node[font=\small, rotate=90] at (-0.5, -4.5) {مذهبی};
    
    % Political groups
    \node[draw, circle, fill=iranred!30, minimum size=1.5cm, font=\tiny, align=center] (left) at (-5, 3) {کمونیست‌ها\\و چپ رادیکال};
    \node[draw, circle, fill=irangold!50, minimum size=2cm, font=\tiny, align=center] (jomhori) at (-2, 3.5) {جمهوری‌خواهان\\سکولار};
    \node[draw, circle, fill=iranblue!40, minimum size=2.5cm, font=\tiny, align=center] (mashroot) at (4, 3) {مشروطه‌خواهان\\پهلوی‌گرایان};
    \node[draw, circle, fill=olgoogreen!50, minimum size=1.8cm, font=\tiny, align=center] (melli) at (-1, -2) {ملی-مذهبی‌ها\\اصلاح‌طلبان};
    \node[draw, circle, fill=empirepurple!30, minimum size=1.5cm, font=\tiny, align=center] (ghomi) at (3, 0) {جنبش‌های\\قومی-ملی};
    \node[draw, circle, fill=noktegray, minimum size=1.2cm, font=\tiny, align=center] (mojahed) at (-4, -3) {مجاهدین\\خلق};
\end{tikzpicture}
\captionof{figure}{نقشه طیف سیاسی اپوزیسیون ایران}
\end{center}

\section{تحلیل هر جریان}

\subsection{مشروطه‌خواهان و پهلوی‌گرایان}

\begin{table}[H]
\centering
\caption{تحلیل SWOT جریان مشروطه‌خواه}
\begin{tabular}{|C{6cm}|C{6cm}|}
\hline
\rowcolor{olgoogreen!30}
\multicolumn{2}{|c|}{\textbf{نقاط قوت (Strengths)}} \\
\hline
\multicolumn{2}{|p{12cm}|}{
\begin{itemize}[rightmargin=1em]
    \item چهره شاخص (رضا پهلوی) با شناخت بالا
    \item پایگاه اجتماعی قابل توجه در طبقه متوسط
    \item ارتباطات بین‌المللی
    \item پیام ساده و قابل فهم
\end{itemize}
} \\
\hline
\rowcolor{iranred!20}
\multicolumn{2}{|c|}{\textbf{نقاط ضعف (Weaknesses)}} \\
\hline
\multicolumn{2}{|p{12cm}|}{
\begin{itemize}[rightmargin=1em]
    \item بار تاریخی رژیم گذشته
    \item فقدان ساختار سازمانی قوی
    \item اختلاف بر سر «پادشاهی» یا «جمهوری»
    \item ضعف ارتباط با داخل کشور
\end{itemize}
} \\
\hline
\rowcolor{iranblue!20}
\multicolumn{2}{|c|}{\textbf{فرصت‌ها (Opportunities)}} \\
\hline
\multicolumn{2}{|p{12cm}|}{
\begin{itemize}[rightmargin=1em]
    \item نوستالژی نسل جوان نسبت به دوره پهلوی
    \item حمایت احتمالی غرب
    \item امکان ائتلاف با جریانات دیگر
\end{itemize}
} \\
\hline
\rowcolor{irangold!30}
\multicolumn{2}{|c|}{\textbf{تهدیدها (Threats)}} \\
\hline
\multicolumn{2}{|p{12cm}|}{
\begin{itemize}[rightmargin=1em]
    \item تبلیغات منفی نظام
    \item مخالفت چپ‌ها و جمهوری‌خواهان
    \item ریسک تفرقه درونی
\end{itemize}
} \\
\hline
\end{tabular}
\end{table}

\begin{noktebox}
\textbf{پایگاه اجتماعی:} طبقه متوسط شهری، نسل جوان حافظه تاریخی ندار، ایرانیان خارج از کشور
\end{noktebox}

\subsection{جمهوری‌خواهان سکولار}

\begin{table}[H]
\centering
\caption{تحلیل SWOT جمهوری‌خواهان سکولار}
\begin{tabular}{|C{6cm}|C{6cm}|}
\hline
\rowcolor{olgoogreen!30}
\textbf{نقاط قوت} & \textbf{نقاط ضعف} \\
\hline
\begin{minipage}{5.8cm}
\begin{itemize}[rightmargin=0.5em, font=\small]
    \item پایبندی به اصول دموکراتیک
    \item کادرهای با تجربه سیاسی
    \item مقبولیت در محافل آکادمیک
\end{itemize}
\end{minipage}
&
\begin{minipage}{5.8cm}
\begin{itemize}[rightmargin=0.5em, font=\small]
    \item پراکندگی و عدم انسجام
    \item فقدان رهبر شاخص
    \item ضعف در بسیج توده‌ای
\end{itemize}
\end{minipage}
\\
\hline
\rowcolor{iranblue!20}
\textbf{فرصت‌ها} & \textbf{تهدیدها} \\
\hline
\begin{minipage}{5.8cm}
\begin{itemize}[rightmargin=0.5em, font=\small]
    \item موضع «نه سلطنت، نه استبداد»
    \item جذب نیروهای میانه
    \item تجربیات بین‌المللی
\end{itemize}
\end{minipage}
&
\begin{minipage}{5.8cm}
\begin{itemize}[rightmargin=0.5em, font=\small]
    \item بین دو صندلی ماندن
    \item فقدان منابع مالی
    \item عدم شناخت عمومی
\end{itemize}
\end{minipage}
\\
\hline
\end{tabular}
\end{table}

\subsection{چپ‌ها و سوسیالیست‌ها}

\begin{itemize}
    \item \textbf{طیف:} از سوسیال‌دموکرات‌ها تا کمونیست‌ها
    \item \textbf{تاریخچه:} سرکوب شدید در دهه ۶۰، بازسازی تدریجی
    \item \textbf{نقاط قوت:} تجربه سازمانی، ارتباط با جنبش کارگری
    \item \textbf{نقاط ضعف:} انگ تاریخی، اختلافات ایدئولوژیک، بار شوروی
\end{itemize}

\subsection{ملی-مذهبی‌ها و اصلاح‌طلبان خارج‌نشین}

\begin{enghelabbox}[title=هشدار]
این جریان پس از رویدادهای دی‌ماه ۱۴۰۴ با بحران هویتی عمیقی مواجه است. بسیاری از چهره‌های شاخص آن به مواضع رادیکال‌تر نزدیک شده‌اند.
\end{enghelabbox}

\begin{itemize}
    \item \textbf{ویژگی:} سابقه درون‌نظامی، تلاش برای اصلاح از درون (شکست‌خورده)
    \item \textbf{چهره‌ها:} برخی اصلاح‌طلبان سابق، ملی-مذهبی‌ها
    \item \textbf{موضع فعلی:} در حال گذار به اپوزیسیون برانداز
\end{itemize}

\subsection{جنبش‌های قومی-ملی}

\begin{table}[H]
\centering
\caption{جنبش‌های قومی-ملی}
\begin{tabular}{|R{2.5cm}|C{2.5cm}|L{7cm}|}
\hline
\rowcolor{empirepurple!20}
\textbf{قومیت} & \textbf{جریانات اصلی} & \textbf{خواسته‌های کلیدی} \\
\hline
کُرد & کومله، دموکرات & فدرالیسم، حقوق فرهنگی \\
\hline
عرب & احوازی‌ها & خودمختاری، حقوق زبانی \\
\hline
بلوچ & جنبش آزادی‌بخش & پایان تبعیض، توسعه \\
\hline
آذری & ملی‌گرایان & حقوق زبانی، هویت \\
\hline
\end{tabular}
\end{table}

\subsection{سازمان مجاهدین خلق}

\begin{enghelabbox}[title=موضع بحث‌برانگیز]
این سازمان به دلیل همکاری با عراق در جنگ و ساختار فرقه‌ای، مورد مناقشه است. اکثر جریانات اپوزیسیون از ائتلاف با آن پرهیز می‌کنند.
\end{enghelabbox}

\section{نقاط همگرایی و واگرایی}

\begin{center}
\begin{tikzpicture}
    % Venn diagram style
    \begin{scope}[blend group=soft light]
        \fill[iranblue!40] (-2,0) circle (3cm);
        \fill[irangold!40] (2,0) circle (3cm);
        \fill[olgoogreen!40] (0,-2.5) circle (3cm);
    \end{scope}
    
    % Labels
    \node[font=\bfseries] at (-3.5, 1) {سکولاریسم};
    \node[font=\bfseries] at (3.5, 1) {دموکراسی};
    \node[font=\bfseries] at (0, -4) {یکپارچگی ایران};
    
    % Center - shared
    \node[font=\small, align=center] at (0, -0.5) {سرنگونی\\نظام};
\end{tikzpicture}
\captionof{figure}{نقاط همگرایی اپوزیسیون}
\end{center}

\subsection{نقاط همگرایی}

\begin{olgoobox}[title=اشتراکات]
\begin{enumerate}
    \item \textbf{سرنگونی نظام:} تقریباً همه جریانات (به جز اصلاح‌طلبان افراطی) موافقند
    \item \textbf{دموکراسی:} توافق کلی بر ضرورت نظام دموکراتیک
    \item \textbf{حقوق بشر:} اجماع بر رعایت حقوق بنیادین
    \item \textbf{جدایی دین از دولت:} توافق قابل توجه
\end{enumerate}
\end{olgoobox}

\subsection{نقاط واگرایی}

\begin{table}[H]
\centering
\caption{موضوعات اختلافی اصلی}
\begin{tabular}{|R{3cm}|L{5cm}|L{5cm}|}
\hline
\rowcolor{iranred!10}
\textbf{موضوع} & \textbf{موضع الف} & \textbf{موضع ب} \\
\hline
شکل حکومت & پادشاهی مشروطه & جمهوری \\
\hline
ساختار کشور & متمرکز & فدرال/خودمختاری \\
\hline
اقتصاد & بازار آزاد & دولت رفاه \\
\hline
عدالت انتقالی & عفو عمومی & محاکمه \\
\hline
رابطه با غرب & همکاری کامل & استقلال نسبی \\
\hline
\end{tabular}
\end{table}

\section{چرا ائتلاف شکل نمی‌گیرد؟}

\begin{center}
\begin{tikzpicture}[
    block/.style={rectangle, rounded corners, draw, fill=enghelabred!30, minimum width=4cm, minimum height=1cm, align=center, font=\small}
]
    \node[block] (ego) at (0, 4) {رقابت شخصی و خودخواهی};
    \node[block] (hist) at (-5, 2) {بار تاریخی و خصومت‌های قدیمی};
    \node[block] (ideo) at (5, 2) {اختلافات ایدئولوژیک عمیق};
    \node[block] (trust) at (-5, 0) {بی‌اعتمادی متقابل};
    \node[block] (struct) at (5, 0) {فقدان ساختار مشترک};
    \node[block] (extern) at (0, -2) {عدم فشار خارجی برای اتحاد};
    
    \node[draw, circle, fill=iranred!50, minimum size=2cm] (center) at (0, 1) {شکست ائتلاف};
    
    \draw[thick, ->] (ego) -- (center);
    \draw[thick, ->] (hist) -- (center);
    \draw[thick, ->] (ideo) -- (center);
    \draw[thick, ->] (trust) -- (center);
    \draw[thick, ->] (struct) -- (center);
    \draw[thick, ->] (extern) -- (center);
\end{tikzpicture}
\captionof{figure}{عوامل شکست ائتلاف‌سازی}
\end{center}

\section{پتانسیل‌های ائتلاف‌سازی}

\begin{empirebox}[title=فرصت‌های پیش رو]
\begin{enumerate}
    \item \textbf{منشور مهسا:} تلاش برای ایجاد پلتفرم مشترک
    \item \textbf{فشار نسل جوان:} خواست عملگرایی به جای ایدئولوژی
    \item \textbf{شکست نظام:} هرچه نظام ضعیف‌تر، انگیزه اتحاد بیشتر
    \item \textbf{مدل حداقلی:} توافق بر اصول پایه بدون حل همه اختلافات
\end{enumerate}
\end{empirebox}

\subsection{مدل پیشنهادی ائتلاف}

\begin{center}
\begin{tikzpicture}
    % Concentric circles
    \draw[thick, fill=iranblue!10] (0,0) circle (4cm);
    \draw[thick, fill=iranblue!20] (0,0) circle (3cm);
    \draw[thick, fill=iranblue!30] (0,0) circle (2cm);
    \draw[thick, fill=iranblue!50] (0,0) circle (1cm);
    
    % Labels
    \node[font=\small\bfseries, align=center] at (0, 0) {اصول\\مشترک};
    \node[font=\tiny, align=center] at (0, 1.5) {ساختار\\هماهنگی};
    \node[font=\tiny, align=center] at (0, 2.5) {جریانات\\اصلی};
    \node[font=\tiny, align=center] at (0, 3.5) {حامیان\\عمومی};
\end{tikzpicture}
\captionof{figure}{مدل دایره‌ای ائتلاف}
\end{center}

\section{توصیه‌های عملی برای هر جریان}

\begin{table}[H]
\centering
\caption{توصیه‌های اختصاصی}
\begin{tabular}{|R{3.5cm}|L{9cm}|}
\hline
\rowcolor{iranblue!20}
\textbf{جریان} & \textbf{توصیه کلیدی} \\
\hline
مشروطه‌خواهان & موضع روشن درباره «پادشاهی یا جمهوری؟» - انعطاف بیشتر \\
\hline
جمهوری‌خواهان & ایجاد ساختار متحد، معرفی چهره شاخص \\
\hline
چپ‌ها & فاصله از انگ تاریخی، تمرکز بر عدالت اجتماعی \\
\hline
ملی-مذهبی‌ها & تعیین تکلیف نهایی با نظام، موضع شفاف \\
\hline
جنبش‌های قومی & تضمین یکپارچگی ایران، مدل فدرالیسم دموکراتیک \\
\hline
\end{tabular}
\end{table}

\begin{kholasebox}[title=خلاصه فصل]
\begin{itemize}
    \item اپوزیسیون ایران طیف متنوعی از جریانات را شامل می‌شود
    \item نقاط همگرایی (سرنگونی نظام، دموکراسی) بیش از واگرایی‌هاست
    \item ائتلاف‌سازی تاکنون به دلایل متعدد شکست خورده
    \item مدل «توافق حداقلی» واقع‌بینانه‌ترین مسیر است
    \item هر جریان نیاز به بازنگری و انعطاف دارد
\end{itemize}
\end{kholasebox}
