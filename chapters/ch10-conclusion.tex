% ch10-conclusion.tex
% فصل دهم: جمع‌بندی

\chapter{جمع‌بندی}
\label{ch:conclusion}

\begin{center}
\begin{tikzpicture}[scale=0.9, transform shape]
    % Sun rising (symbolic of freedom)
    \fill[irangold!30] (0, 0) circle (3cm);
    \foreach \a in {0, 30, ..., 330} {
        \draw[ultra thick, irangold, -{Stealth}] (0,0) -- (\a:4);
    }
    
    % Text
    \node[concept, scale=1.2, fill=white, draw=iranblue] at (0,0) {آزادی و\\دموکراسی};
    
    \node[below=4.5cm of goal, align=center, font= \small \color{iranblue!70}] at (0,0) {افق روشن ایران فردا};
\end{tikzpicture}
\captionof{figure}{پایان استبداد و طلوع آزادی}
\end{center}

\section{ده پیام کلیدی}

\begin{enumerate}
    \item \textbf{نظام در بحران است:} بحران مشروعیت، اقتصاد و انزوای بین‌المللی
    
    \item \textbf{سرنگونی ممکن است:} اما نیازمند استراتژی و سازماندهی
    
    \item \textbf{ائتلاف ضروری است:} تفرقه بزرگ‌ترین ضعف اپوزیسیون است
    
    \item \textbf{خشونت‌پرهیزی برتری دارد:} تجربه نشان می‌دهد مؤثرتر است
    
    \item \textbf{روز بعد مهم‌تر است:} آمادگی برای حکومت انتقالی حیاتی است
    
    \item \textbf{ریسک‌ها جدی‌اند:} جنگ داخلی و تجزیه باید اجتناب شود
    
    \item \textbf{بین‌الملل ابزار است:} نه ناجی - استفاده هوشمند
    
    \item \textbf{نسل جوان کلید است:} انرژی و شجاعت بی‌نظیر
    
    \item \textbf{زمان مهم است:} فرصت ابدی نیست
    
    \item \textbf{امید واقعی وجود دارد:} تغییر آمدنی است
\end{enumerate}

\section{فراخوان عمل}

\begin{empirebox}[title=به رهبران اپوزیسیون]
\begin{itemize}
    \item منافع شخصی را کنار بگذارید
    \item ائتلاف را جدی بگیرید
    \item ساختار بسازید
    \item با داخل ارتباط برقرار کنید
\end{itemize}
\end{empirebox}

\begin{olgoobox}[title=به فعالان داخل]
\begin{itemize}
    \item امنیت شخصی اولویت است
    \item شبکه‌سازی کنید
    \item امید را زنده نگه دارید
    \item منتظر لحظه مناسب باشید
\end{itemize}
\end{olgoobox}

\begin{noktebox}
\textbf{به دیاسپورا:} صدای داخل باشید. لابی کنید. منابع فراهم کنید.
\end{noktebox}

\section{کلام آخر: امید واقع‌بینانه}

\begin{naghlbox}
تاریخ به ما می‌آموزد که هیچ نظام استبدادی ابدی نیست. دیوار برلین فرو ریخت. آپارتاید پایان یافت. شوروی فروپاشید. جمهوری اسلامی نیز استثنا نخواهد بود.

اما تغییر خودبه‌خود نمی‌آید. تغییر نیازمند:
\begin{itemize}
    \item \textbf{وحدت} نیروهای آزادی‌خواه
    \item \textbf{صبر} استراتژیک
    \item \textbf{آمادگی} برای بهره‌برداری از فرصت‌ها
    \item \textbf{شجاعت} مردم
\end{itemize}

آینده در دست ماست. وقت عمل است.
\end{naghlbox}

\vspace{2cm}

\begin{center}
\textit{«پایان استبداد، آغاز آزادی»}
\end{center}
