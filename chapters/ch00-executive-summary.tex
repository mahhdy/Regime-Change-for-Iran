% ch00-executive-summary.tex
% فصل صفر: خلاصه مدیریتی

\chapter*{خلاصه مدیریتی}
\addcontentsline{toc}{chapter}{خلاصه مدیریتی}

\begin{center}
\begin{tikzpicture}[node distance=1.5cm, scale=0.85, transform shape]
    % Background box
    \draw[background_rect] (-6, 2) rectangle (6, -6);
    
    % Main Goal
    \node[concept] (goal) at (0,0) {گذار به\\دموکراسی};
    
    % Pillars
    \node[step] (p1) at (-4, 1) {\keyword{ائتلاف سیاسی}};
    \node[step] (p2) at (4, 1) {\keyword{بسیج مردمی}};
    \node[step] (p3) at (-4, -4) {\keyword{حمایت جهانی}};
    \node[step] (p4) at (4, -4) {\keyword{مدیریت بحران}};
    
    % Connections
    \draw[connection] (goal) -- (p1);
    \draw[connection] (goal) -- (p2);
    \draw[connection] (goal) -- (p3);
    \draw[connection] (goal) -- (p4);
    
    % Icons/Symbols (Simplified representation)
    \node[icon] at (-4, 1.8) {🤝};
    \node[icon] at (4, 1.8) {✊};
    \node[icon] at (-4, -4.8) {🌐};
    \node[icon] at (4, -4.8) {🛡️};
    
    \node[below=2.5cm of goal, align=center, font= \itshape \color{iranblue!70}] {ساختار چهارگانه استراتژی گذار};
\end{tikzpicture}
\captionof{figure}{ارکان اصلی استراتژی گذار}
\end{center}

\section*{چکیده وضعیت}

ایران در نقطه عطف تاریخی قرار دارد. رویدادهای دی‌ماه ۱۴۰۴ نشان داد که نظام جمهوری اسلامی به مرحله‌ای از بحران رسیده که بازگشت به وضعیت سابق را دشوار می‌سازد. این راهنما تلاش دارد نقشه راهی واقع‌بینانه برای گذار به دموکراسی ارائه دهد.

\begin{kholasebox}[title=یافته‌های کلیدی]
\begin{itemize}
    \item نظام دچار \keyword{بحران مشروعیت} عمیق است
    \item ظرفیت سرکوب همچنان بالاست اما \keyword{هزینه‌های آن رو به افزایش} است
    \item اپوزیسیون پراکنده اما \keyword{پتانسیل ائتلاف} دارد
    \item شرایط بین‌المللی نسبتاً \keyword{مساعد} است
    \item ریسک‌های جدی (جنگ داخلی، تجزیه) باید مدیریت شوند
\end{itemize}
\end{kholasebox}

\newpage

\section*{ماتریس سناریوهای گذار}

\begin{table}[H]
\centering
\caption{ماتریس مقایسه‌ای پنج سناریوی اصلی گذار}
\begin{tabular}{|R{2.5cm}|C{2cm}|C{2cm}|C{2cm}|C{2.5cm}|}
\hline
\rowcolor{iranblue!20}
\textbf{سناریو} & \textbf{احتمال} & \textbf{هزینه} & \textbf{مدت} & \textbf{ریسک اصلی} \\
\hline
انقلاب خشونت‌پرهیز & متوسط (۳۵٪) & متوسط & ۶-۱۸ ماه & سرکوب خونین \\
\hline
مقاومت مسلحانه & پایین (۱۵٪) & بسیار بالا & نامشخص & جنگ داخلی \\
\hline
کودتای نظامی & پایین (۱۰٪) & متوسط & سریع & دیکتاتوری جدید \\
\hline
فروپاشی از درون & متوسط (۲۵٪) & بالا & متغیر & هرج‌ومرج \\
\hline
گذار مذاکره‌ای & پایین (۱۵٪) & پایین & طولانی & تداوم استبداد \\
\hline
\end{tabular}
\end{table}

\section*{نمودار تصمیم‌گیری استراتژیک}

\begin{center}
\begin{tikzpicture}[node distance=2cm, scale=0.9, transform shape]
    % Starting point
    \node[block, fill=iranblue!30] (start) {وضعیت کنونی};
    
    % Decision nodes
    \node[decision, below=of start] (d1) {آیا ائتلاف ممکن است؟};
    \node[process, below left=2cm and 2cm of d1] (eattelaf) {تشکیل ائتلاف فراگیر};
    \node[process, below right=2cm and 2cm of d1] (taksar) {ادامه فعالیت مستقل};
    
    % Second level
    \node[decision, below=of eattelaf] (d2) {آیا بسیج کافی است؟};
    \node[block, fill=olgoogreen, below left=of d2] (action) {اقدام هماهنگ};
    \node[process, below right=of d2] (build) {ظرفیت‌سازی};
    
    % Arrows
    \draw[arrow] (start) -- (d1);
    \draw[arrow] (d1) -- node[above left] {بله} (eattelaf);
    \draw[arrow] (d1) -- node[above right] {خیر} (taksar);
    \draw[arrow] (eattelaf) -- (d2);
    \draw[arrow] (d2) -- node[above left] {بله} (action);
    \draw[arrow] (d2) -- node[above right] {خیر} (build);
    \draw[arrow] (build.south) |- ++(-3,-1) |- (d2.east);
\end{tikzpicture}
\end{center}

\newpage

\section*{توصیه‌های فوری ده‌گانه}

\begin{empirebox}[title=اقدامات فوری برای اپوزیسیون]
\begin{enumerate}
    \item \textbf{ائتلاف‌سازی:} آغاز مذاکرات جدی بین جریان‌های اصلی بدون پیش‌شرط
    \item \textbf{ساختارسازی:} ایجاد نهادهای هماهنگ‌کننده با ساختار شفاف
    \item \textbf{ارتباط با داخل:} تقویت کانال‌های امن ارتباطی با فعالان داخل
    \item \textbf{دیپلماسی:} حضور منسجم در محافل بین‌المللی
    \item \textbf{رسانه:} هماهنگی پیام‌رسانی و مقابله با جنگ روانی نظام
    \item \textbf{منابع:} جذب و مدیریت شفاف منابع مالی
    \item \textbf{آموزش:} تربیت کادرهای مدیریتی برای دوره گذار
    \item \textbf{مستندسازی:} ثبت جنایات برای عدالت انتقالی
    \item \textbf{برنامه‌ریزی:} تدوین طرح حکمرانی انتقالی
    \item \textbf{شبکه‌سازی:} ارتباط با نخبگان ایرانی در سراسر جهان
\end{enumerate}
\end{empirebox}

\section*{پیام کلیدی}

\begin{naghlbox}
«گذار موفق نیازمند \textbf{اتحاد}، \textbf{صبر استراتژیک} و \textbf{آمادگی سازمانی} است. تجربه تاریخی نشان می‌دهد که رژیم‌ها نه در اوج سرکوب، بلکه زمانی سقوط می‌کنند که مردم \textbf{امید واقع‌بینانه} و \textbf{آلترناتیو معتبر} ببینند.»
\end{naghlbox}

\begin{center}
\begin{tikzpicture}
    % Radar chart for scenario comparison
    \begin{axis}[
        width=10cm,
        height=8cm,
        xbar,
        xlabel={امتیاز (از ۱۰)},
        symbolic y coords={ریسک,هزینه,احتمال موفقیت,سرعت,پایداری نتیجه},
        ytick=data,
        nodes near coords,
        nodes near coords align={horizontal},
        bar width=15pt,
        y dir=reverse,
    ]
    \addplot coordinates {(7,ریسک) (6,هزینه) (6,احتمال موفقیت) (5,سرعت) (7,پایداری نتیجه)};
    \end{axis}
\end{tikzpicture}
\captionof{figure}{ارزیابی کلی سناریوی ترکیبی (واقع‌بینانه‌ترین مسیر)}
\end{center}

\newpage

\section*{مخاطبان این راهنما}

\begin{table}[H]
\centering
\begin{tabular}{|R{3cm}|L{9cm}|}
\hline
\rowcolor{iranblue!20}
\textbf{مخاطب} & \textbf{توصیه کلیدی} \\
\hline
رهبران اپوزیسیون & فصول ۴، ۶ و ۹ را با دقت بخوانید. اولویت: ائتلاف‌سازی \\
\hline
فعالان داخل کشور & فصول ۲، ۶ و ۸ را مطالعه کنید. امنیت شخصی را جدی بگیرید \\
\hline
دیپلمات‌ها و سیاستگذاران & فصول ۵ و ۷ مرجع اصلی شماست \\
\hline
پژوهشگران & پیوست‌ها منابع غنی نظری و تجربی ارائه می‌دهند \\
\hline
شهروندان عادی & فصل ۱ و خلاصه‌های هر فصل کافی است \\
\hline
\end{tabular}
\caption{راهنمای استفاده برای مخاطبان مختلف}
\end{table}
