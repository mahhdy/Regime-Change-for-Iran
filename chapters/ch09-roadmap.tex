% ch09-roadmap.tex
% فصل نهم: نقشه راه عملیاتی

\chapter{نقشه راه عملیاتی}
\label{ch:roadmap}

\begin{center}
\begin{tikzpicture}[scale=0.9, transform shape]
    % Roadmap Steps
    \foreach \i/\label in {1/\rl{تثبیت}, 2/\rl{ائتلاف}, 3/\rl{بسیج}, 4/\rl{اقدام}} {
        \node[step, minimum width=2.5cm, fill=iranblue!\the\numexpr\i*20\relax] (s\i) at (\i*3, 0) {\label};
    }
    \draw[connection] (s1) -- (s2);
    \draw[connection] (s2) -- (s3);
    \draw[connection] (s3) -- (s4);
    
    \node[below=1.5cm of s2.south east, align=center, font= \small \color{iranblue!70}] {\rl{توالی زمانی مراحل عملیاتی}};
\end{tikzpicture}
\captionof{figure}{\rl{نقشه راه کلی عملیات گذار}}
\end{center}

\begin{naghlbox}
«هدف بدون برنامه فقط یک آرزوست. برنامه بدون عمل فقط یک رویاست.»
\end{naghlbox}

\section{مرحله اول: تثبیت (۰-۳ ماه)}

\begin{table}[H]
\centering
\begin{tabular}{|C{1cm}|R{10cm}|C{3cm}|}
\hline
\rowcolor{iranblue!20}
& \textbf{\rl{اقدام}} & \textbf{\rl{مسئول}} \\
\hline
$\square$ & \rl{تشکیل کمیته هماهنگی موقت} & \rl{رهبران} \\
\hline
$\square$ & \rl{تعیین سخنگوی واحد} & \rl{کمیته} \\
\hline
$\square$ & \rl{راه‌اندازی کانال‌های ارتباطی امن} & \rl{فنی} \\
\hline
$\square$ & \rl{تدوین بیانیه اصول مشترک} & \rl{همه} \\
\hline
\end{tabular}
\caption{\rl{چک‌لیست مرحله تثبیت}}
\end{table}

\section{مرحله دوم: ائتلاف‌سازی (۳-۱۲ ماه)}

\begin{center}
\begin{tikzpicture}[node distance=2.5cm]
    \node[block, fill=iranblue!20] (talks) {\rl{مذاکرات دوجانبه}};
    \node[block, fill=iranblue!30, right=of talks] (doc) {\rl{تدوین منشور}};
    \node[block, fill=iranblue!40, right=of doc] (sign) {\rl{امضای منشور}};
    \node[block, fill=olgoogreen!50, right=of sign] (announce) {\rl{اعلام عمومی}};
    
    \draw[arrow] (talks) -- (doc);
    \draw[arrow] (doc) -- (sign);
    \draw[arrow] (sign) -- (announce);
\end{tikzpicture}
\captionof{figure}{\rl{فرآیند ائتلاف‌سازی}}
\end{center}

\section{مرحله سوم: بسیج (۶-۱۸ ماه)}

\begin{itemize}
    \item ارتباط با شبکه‌های داخلی
    \item کمپین‌های رسانه‌ای
    \item لابی‌گری بین‌المللی
    \item جذب منابع
    \item آموزش کادرها
\end{itemize}

\section{مرحله چهارم: اقدام}

\begin{empirebox}[title=شرایط آغاز]
\begin{enumerate}
    \item ائتلاف پایدار شکل گرفته
    \item حمایت بین‌المللی تأمین شده
    \item ظرفیت بسیج کافی
    \item رویداد آغازگر (Trigger)
    \item آمادگی برای حکومت انتقالی
\end{enumerate}
\end{empirebox}

\section{شاخص‌های پایش پیشرفت}

\begin{table}[H]
\centering
\begin{tabular}{|R{4cm}|C{3cm}|R{8cm}|}
\hline
\rowcolor{irangold!20}
\textbf{\rl{شاخص}} & \textbf{\rl{هدف}} & \textbf{\rl{وضعیت فعلی}} \\
\hline
\rl{تعداد جریانات در ائتلاف} & \rl{۵+} & \rl{در حال مذاکره} \\
\hline
\rl{بودجه سالانه} & \rl{۱۰M+} & \rl{نامشخص} \\
\hline
\rl{حمایت دولت‌ها} & \rl{۳+} & \rl{۱-۲} \\
\hline
\rl{ارتباط با داخل} & \rl{فعال} & \rl{محدود} \\
\hline
\end{tabular}
\caption{\rl{داشبورد پایش}}
\end{table}

\section{نقاط تصمیم‌گیری کلیدی}

\begin{center}
\begin{tikzpicture}[node distance=2cm]
    \node[decision] (d1) {\rl{ائتلاف ممکن؟}};
    \node[block, right=of d1] (continue) {\rl{ادامه مستقل}};
    \node[block, below=of d1] (coalition) {\rl{تشکیل ائتلاف}};
    
    \node[decision, below=of coalition] (d2) {\rl{بسیج کافی؟}};
    \node[block, right=of d2] (build) {\rl{ظرفیت‌سازی}};
    \node[block, below=of d2] (action) {\rl{اقدام}};
    
    \draw[arrow] (d1) -- node[above] {\rl{خیر}} (continue);
    \draw[arrow] (d1) -- node[right] {\rl{بله}} (coalition);
    \draw[arrow] (coalition) -- (d2);
    \draw[arrow] (d2) -- node[above] {\rl{خیر}} (build);
    \draw[arrow] (d2) -- node[right] {\rl{بله}} (action);
    \draw[arrow] (build.south) |- ++(0,-0.5) -| (d2.east);
\end{tikzpicture}
\captionof{figure}{\rl{درخت تصمیم‌گیری}}
\end{center}

\begin{kholasebox}[title=خلاصه فصل]
\begin{itemize}
    \item نقشه راه چهار مرحله‌ای: تثبیت، ائتلاف، بسیج، اقدام
    \item هر مرحله اقدامات و مسئولین مشخص دارد
    \item شاخص‌های پایش باید مرتب بررسی شوند
    \item نقاط تصمیم‌گیری کلیدی را بشناسید
\end{itemize}
\end{kholasebox}
