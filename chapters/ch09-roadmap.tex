% ch09-roadmap.tex
% فصل نهم: نقشه راه عملیاتی

\chapter{نقشه راه عملیاتی}
\label{ch:roadmap}

\begin{center}
\begin{tikzpicture}[scale=0.9, transform shape]
    % Roadmap Steps
    \foreach \i/\label in {1/تثبیت, 2/ائتلاف, 3/بسیج, 4/اقدام} {
        \node[step, minimum width=2.5cm, fill=iranblue!\the\numexpr\i*20\relax] (s\i) at (\i*3, 0) {\label};
    }
    \draw[connection] (s1) -- (s2);
    \draw[connection] (s2) -- (s3);
    \draw[connection] (s3) -- (s4);
    
    \node[below=1.5cm of s2.south east, align=center, font= \small \color{iranblue!70}] {توالی زمانی مراحل عملیاتی};
\end{tikzpicture}
\captionof{figure}{نقشه راه کلی عملیات گذار}
\end{center}

\begin{naghlbox}
«هدف بدون برنامه فقط یک آرزوست. برنامه بدون عمل فقط یک رویاست.»
\end{naghlbox}

\section{مرحله اول: تثبیت (۰-۳ ماه)}

\begin{table}[H]
\centering
\begin{tabular}{|C{0.5cm}|L{8cm}|C{2cm}|}
\hline
\rowcolor{iranblue!20}
& \textbf{اقدام} & \textbf{مسئول} \\
\hline
$\square$ & تشکیل کمیته هماهنگی موقت & رهبران \\
\hline
$\square$ & تعیین سخنگوی واحد & کمیته \\
\hline
$\square$ & راه‌اندازی کانال‌های ارتباطی امن & فنی \\
\hline
$\square$ & تدوین بیانیه اصول مشترک & همه \\
\hline
\end{tabular}
\caption{چک‌لیست مرحله تثبیت}
\end{table}

\section{مرحله دوم: ائتلاف‌سازی (۳-۱۲ ماه)}

\begin{center}
\begin{tikzpicture}[node distance=2.5cm]
    \node[block, fill=iranblue!20] (talks) {مذاکرات دوجانبه};
    \node[block, fill=iranblue!30, right=of talks] (doc) {تدوین منشور};
    \node[block, fill=iranblue!40, right=of doc] (sign) {امضای منشور};
    \node[block, fill=olgoogreen!50, right=of sign] (announce) {اعلام عمومی};
    
    \draw[arrow] (talks) -- (doc);
    \draw[arrow] (doc) -- (sign);
    \draw[arrow] (sign) -- (announce);
\end{tikzpicture}
\captionof{figure}{فرآیند ائتلاف‌سازی}
\end{center}

\section{مرحله سوم: بسیج (۶-۱۸ ماه)}

\begin{itemize}
    \item ارتباط با شبکه‌های داخلی
    \item کمپین‌های رسانه‌ای
    \item لابی‌گری بین‌المللی
    \item جذب منابع
    \item آموزش کادرها
\end{itemize}

\section{مرحله چهارم: اقدام}

\begin{empirebox}[title=شرایط آغاز]
\begin{enumerate}
    \item ائتلاف پایدار شکل گرفته
    \item حمایت بین‌المللی تأمین شده
    \item ظرفیت بسیج کافی
    \item رویداد آغازگر (Trigger)
    \item آمادگی برای حکومت انتقالی
\end{enumerate}
\end{empirebox}

\section{شاخص‌های پایش پیشرفت}

\begin{table}[H]
\centering
\begin{tabular}{|R{4cm}|C{2cm}|L{5cm}|}
\hline
\rowcolor{irangold!20}
\textbf{شاخص} & \textbf{هدف} & \textbf{وضعیت فعلی} \\
\hline
تعداد جریانات در ائتلاف & ۵+ & در حال مذاکره \\
\hline
بودجه سالانه & ۱۰M+ & نامشخص \\
\hline
حمایت دولت‌ها & ۳+ & ۱-۲ \\
\hline
ارتباط با داخل & فعال & محدود \\
\hline
\end{tabular}
\caption{داشبورد پایش}
\end{table}

\section{نقاط تصمیم‌گیری کلیدی}

\begin{center}
\begin{tikzpicture}[node distance=2cm]
    \node[decision] (d1) {ائتلاف ممکن؟};
    \node[block, right=of d1] (continue) {ادامه مستقل};
    \node[block, below=of d1] (coalition) {تشکیل ائتلاف};
    
    \node[decision, below=of coalition] (d2) {بسیج کافی؟};
    \node[block, right=of d2] (build) {ظرفیت‌سازی};
    \node[block, below=of d2] (action) {اقدام};
    
    \draw[arrow] (d1) -- node[above] {خیر} (continue);
    \draw[arrow] (d1) -- node[right] {بله} (coalition);
    \draw[arrow] (coalition) -- (d2);
    \draw[arrow] (d2) -- node[above] {خیر} (build);
    \draw[arrow] (d2) -- node[right] {بله} (action);
    \draw[arrow] (build.south) |- ++(0,-0.5) -| (d2.east);
\end{tikzpicture}
\captionof{figure}{درخت تصمیم‌گیری}
\end{center}

\begin{kholasebox}[title=خلاصه فصل]
\begin{itemize}
    \item نقشه راه چهار مرحله‌ای: تثبیت، ائتلاف، بسیج، اقدام
    \item هر مرحله اقدامات و مسئولین مشخص دارد
    \item شاخص‌های پایش باید مرتب بررسی شوند
    \item نقاط تصمیم‌گیری کلیدی را بشناسید
\end{itemize}
\end{kholasebox}
