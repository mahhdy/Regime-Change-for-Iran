% ch02-regime-analysis.tex
% فصل دوم: تشریح نظام حاکم

\chapter{تشریح نظام حاکم}
\label{ch:regime-analysis}

\begin{center}
\begin{tikzpicture}[scale=0.9, transform shape]
    % Tower of Power
    \draw[thick, iranblue, fill=iranblue!10] (-3, 0) -- (3, 0) -- (1, 6) -- (-1, 6) -- cycle;
    
    % Core levels
    \node[rectangle, draw=iranred, fill=iranred!10, minimum width=1.5cm, minimum height=0.8cm] at (0, 5.2) {\rl{رهبری}};
    \node[rectangle, draw=iranblue, fill=iranblue!20, minimum width=3cm, minimum height=0.8cm] at (0, 4) {\rl{سپاه و بسیج}};
    \node[rectangle, draw=irangold, fill=irangold!20, minimum width=4.5cm, minimum height=0.8cm] at (0, 2.5) {\rl{اقتصاد و رانت}};
    \node[rectangle, draw=irangreen, fill=irangreen!20, minimum width=5.5cm, minimum height=0.8cm] at (0, 1) {\rl{ایدئولوژی}};
    
    % Pillars holding it up (symbolic)
    \draw[ultra thick, gray] (-2.5, -1) -- (-2.5, 0);
    \draw[ultra thick, gray] (2.5, -1) -- (2.5, 0);
    \draw[ultra thick, gray] (0, -1) -- (0, 0);
    
    \node[below=1.2cm of goal, align=center, font= \small \color{iranred!70}] at (0,-1) {\rl{ستون‌های بقای رژیم}};
\end{tikzpicture}
\captionof{figure}{\rl{ساختار عمودی قدرت در رژیم}}
\end{center}

\vspace{0.5cm}

\begin{naghlbox}
«برای شکست دادن دشمن، ابتدا باید او را شناخت. هر نقطه قوت، نقطه ضعفی پنهان دارد.»
\end{naghlbox}

\section{ماهیت نظام: شبه‌تمامیت‌خواه}

جمهوری اسلامی را نمی‌توان در دسته‌بندی‌های سنتی جای داد. این نظام ترکیبی از عناصر مختلف است:

\begin{center}
\begin{tikzpicture}[
    every node/.style={font=\small},
    box/.style={rectangle, rounded corners, draw, minimum width=3cm, minimum height=1cm, align=center}
]
    \node[box, fill=iranblue!20] (center) at (0,0) {\rl{جمهوری اسلامی}};
    
    \node[box, fill=iranred!20] (theo) at (-4, 2) {\rl{تئوکراسی}};
    \node[box, fill=irangold!20] (auth) at (0, 3) {\rl{اقتدارگرایی}};
    \node[box, fill=empirepurple!20] (total) at (4, 2) {\rl{شبه‌تمامیت‌خواهی}};
    \node[box, fill=olgoogreen!20] (klept) at (-4, -2) {\rl{کلپتوکراسی}};
    \node[box, fill=noktegray] (preto) at (4, -2) {\rl{پرتوریانیسم}};
    
    \draw[thick, ->] (center) -- (theo);
    \draw[thick, ->] (center) -- (auth);
    \draw[thick, ->] (center) -- (total);
    \draw[thick, ->] (center) -- (klept);
    \draw[thick, ->] (center) -- (preto);
\end{tikzpicture}
\captionof{figure}{\rl{ترکیب عناصر نظام جمهوری اسلامی}}
\end{center}

\subsection{ویژگی‌های کلیدی}

\begin{table}[H]
\centering
\caption{مشخصات نظام جمهوری اسلامی}
\begin{tabular}{|R{4cm}|R{11cm}|}
\hline
\rowcolor{iranblue!20}
\textbf{\rl{ویژگی}} & \textbf{\rl{توضیح}} \\
\hline
\rl{ولایت فقیه} & \rl{رهبر دارای قدرت نامحدود و غیرپاسخگو} \\
\hline
\rl{دوگانگی قدرت} & \rl{نهادهای موازی انتخابی و انتصابی} \\
\hline
\rl{ایدئولوژی مذهبی} & \rl{اسلام سیاسی شیعی} \\
\hline
\rl{اقتصاد رانتی} & \rl{وابستگی به نفت و کنترل دولتی} \\
\hline
\rl{سرکوب نظام‌یافته} & \rl{دستگاه امنیتی گسترده} \\
\hline
\end{tabular}
\end{table}

\section{ستون‌های قدرت نظام}

\begin{empirebox}[title=چارچوب تحلیلی]
طبق نظریه جین شارپ، هر نظام استبدادی بر تعدادی «ستون قدرت» استوار است. تضعیف این ستون‌ها کلید سرنگونی نظام است.
\end{empirebox}

\subsection{ستون نظامی-امنیتی}

\begin{center}
\begin{tikzpicture}[
    org/.style={rectangle, rounded corners, draw=iranblue, fill=kholaseblue, minimum width=2.5cm, minimum height=0.8cm, font=\small, align=center},
    level 1/.style={sibling distance=4cm},
    level 2/.style={sibling distance=2.5cm},
    edge from parent/.style={draw, thick}
]
    \node[org, fill=iranblue!40] {\rl{رهبری}}
        child { node[org] {\rl{سپاه پاسداران}}
            child { node[org, font=\tiny] {\rl{نیروی قدس}} }
            child { node[org, font=\tiny] {\rl{بسیج}} }
        }
        child { node[org] {\rl{ارتش}}
            child { node[org, font=\tiny] {\rl{نیروی زمینی}} }
            child { node[org, font=\tiny] {\rl{نیروی هوایی}} }
        }
        child { node[org] {\rl{اطلاعات}}
            child { node[org, font=\tiny] {\rl{وزارت اطلاعات}} }
            child { node[org, font=\tiny] {\rl{اطلاعات سپاه}} }
        };
\end{tikzpicture}
\captionof{figure}{\rl{ساختار نیروهای نظامی-امنیتی}}
\end{center}

\subsubsection{سپاه پاسداران انقلاب اسلامی}

\begin{itemize}
    \item \textbf{نفرات:} حدود ۱۹۰,۰۰۰ نفر (بدون بسیج)
    \item \textbf{بودجه:} تخمین ۲۰-۳۰ میلیارد دلار سالانه
    \item \textbf{امپراتوری اقتصادی:} کنترل بخش بزرگی از اقتصاد
    \item \textbf{وابستگی به نظام:} کامل - سرنوشت سپاه به بقای نظام گره خورده
\end{itemize}

\subsubsection{بسیج مستضعفین}

\begin{itemize}
    \item \textbf{عضویت رسمی:} ۲۰ میلیون نفر (ادعای رسمی)
    \item \textbf{نیروی فعال:} تخمین ۵۰۰,۰۰۰-۱,۰۰۰,۰۰۰ نفر
    \item \textbf{نقش:} سرکوب داخلی، جاسوسی اجتماعی
    \item \textbf{انگیزه:} ترکیبی از ایدئولوژی و منافع مادی
\end{itemize}

\subsection{ستون ایدئولوژیک-مذهبی}

\begin{table}[H]
\centering
\begin{tabular}{|R{4cm}|C{3cm}|R{8cm}|}
\hline
\rowcolor{irangold!30}
\textbf{\rl{نهاد}} & \textbf{\rl{بودجه سالانه}} & \textbf{\rl{نقش}} \\
\hline
\rl{حوزه‌های علمیه} & \rl{۲ میلیارد دلار} & \rl{تولید کادر مذهبی-سیاسی} \\
\hline
\rl{صدا و سیما} & \rl{۱ میلیارد دلار} & \rl{تبلیغات و کنترل اطلاعات} \\
\hline
\rl{نهاد تبلیغات} & \rl{۵۰۰ میلیون دلار} & \rl{گسترش ایدئولوژی} \\
\hline
\rl{وزارت ارشاد} & \rl{۳۰۰ میلیون دلار} & \rl{سانسور و کنترل فرهنگی} \\
\hline
\end{tabular}
\caption{نهادهای ایدئولوژیک و بودجه تخمینی}
\end{table}

\subsection{ستون اقتصادی-رانتی}

\begin{center}
\begin{tikzpicture}
    \begin{axis}[
        ybar,
        ylabel={\rl{درصد از GDP}},
        symbolic x coords={\rl{نفت و گاز}, \rl{سپاه و بنیادها}, \rl{بخش خصوصی}, \rl{بخش دولتی}},
        xtick=data,
        nodes near coords,
        nodes near coords align={vertical},
        ymin=0,
        ymax=50,
        bar width=25pt,
        x tick label style={rotate=45, anchor=east, font=\small},
        width=12cm,
        height=8cm
    ]
    \addplot[fill=iranblue!50] coordinates {
        (\rl{نفت و گاز}, 40)
        (\rl{سپاه و بنیادها}, 25)
        (\rl{بخش خصوصی}, 20)
        (\rl{بخش دولتی}, 15)
    };
    \end{axis}
\end{tikzpicture}
\captionof{figure}{\rl{توزیع تخمینی اقتصاد ایران}}
\end{center}

\subsection{شبکه‌های وابستگی منطقه‌ای}

\begin{table}[H]
\centering
\caption{نیروهای نیابتی جمهوری اسلامی}
\begin{tabular}{|R{4cm}|C{3cm}|C{4cm}|R{4cm}|}
\hline
\rowcolor{iranred!20}
\textbf{\rl{گروه}} & \textbf{\rl{کشور}} & \textbf{\rl{نفرات تخمینی}} & \textbf{\rl{وضعیت فعلی}} \\
\hline
\rl{حزب‌الله لبنان} & \rl{لبنان} & \rl{۴۵,۰۰۰} & \rl{تضعیف شدید} \\
\hline
\rl{حشد الشعبی} & \rl{عراق} & \rl{۱۵۰,۰۰۰} & \rl{تحت فشار} \\
\hline
\rl{حوثی‌ها} & \rl{یمن} & \rl{۲۰۰,۰۰۰} & \rl{فعال} \\
\hline
\rl{گروه‌های فلسطینی} & \rl{غزه/...} & \rl{متغیر} & \rl{تضعیف شده} \\
\hline
\end{tabular}
\end{table}

\section{نقاط ضعف و شکاف‌ها}

\begin{enghelabbox}[title=نقطه کلیدی]
هر نظامی، هرقدر هم قدرتمند، دارای نقاط ضعف است. شناسایی و بهره‌برداری از این نقاط، اساس استراتژی گذار است.
\end{enghelabbox}

\subsection{بحران مشروعیت}

\begin{center}
\begin{tikzpicture}
    \begin{axis}[
        xlabel={\rl{سال}},
        ylabel={\rl{درصد رضایت (تخمینی)}},
        xmin=1378, xmax=1404,
        ymin=0, ymax=100,
        xtick={1378, 1388, 1396, 1401, 1404},
        xticklabels={۱۳۷۸, ۱۳۸۸, ۱۳۹۶, ۱۴۰۱, ۱۴۰۴},
        width=12cm,
        height=7cm,
        grid=major,
        legend pos=north east
    ]
    \addplot[thick, color=iranblue, mark=*] coordinates {
        (1378, 65)
        (1388, 45)
        (1396, 30)
        (1401, 15)
        (1404, 8)
    };
    \legend{\rl{رضایت از نظام}}
    \end{axis}
\end{tikzpicture}
\captionof{figure}{\rl{روند نزولی مشروعیت نظام (تخمینی)}}
\end{center}

\subsection{بحران اقتصادی}

\begin{itemize}
    \item \textbf{تورم:} بیش از ۵۰٪ سالانه
    \item \textbf{بیکاری:} رسمی ۱۲٪، واقعی ۲۵-۳۰٪
    \item \textbf{فقر:} بیش از ۳۰٪ زیر خط فقر
    \item \textbf{ارزش پول ملی:} سقوط ۹۰٪ در دهه گذشته
\end{itemize}

\subsection{شکاف نسلی}

\begin{noktebox}
بیش از ۶۰٪ جمعیت ایران زیر ۴۰ سال سن دارند. این نسل نه انقلاب ۵۷ را به یاد دارد و نه تعلقی به آرمان‌های نظام.
\end{noktebox}

\subsection{شکاف مرکز-پیرامون}

\begin{table}[H]
\centering
\begin{tabular}{|R{4cm}|C{3cm}|R{8cm}|}
\hline
\rowcolor{olgoogreen!30}
\textbf{\rl{منطقه}} & \textbf{\rl{جمعیت تخمینی}} & \textbf{\rl{موضوع اصلی}} \\
\hline
\rl{کردستان} & \rl{۱۰ میلیون} & \rl{خودمختاری، سرکوب فرهنگی} \\
\hline
\rl{بلوچستان} & \rl{۳ میلیون} & \rl{فقر، تبعیض، سرکوب} \\
\hline
\rl{خوزستان} & \rl{۵ میلیون} & \rl{آب، محیط زیست، تبعیض} \\
\hline
\rl{آذربایجان} & \rl{۱۵ میلیون} & \rl{حقوق زبانی، توسعه} \\
\hline
\end{tabular}
\caption{شکاف‌های منطقه‌ای}
\end{table}

\subsection{تنش‌های درون‌حکومتی}

\begin{center}
\begin{tikzpicture}[
    faction/.style={ellipse, draw, minimum width=3cm, minimum height=1.5cm, align=center, font=\small}
]
    \node[faction, fill=iranblue!30] (asl) at (0, 2) {\rl{اصول‌گرایان سنتی}};
    \node[faction, fill=iranred!30] (tanad) at (4, 0) {\rl{تندروها (پایداری)}};
    \node[faction, fill=olgoogreen!30] (sepah) at (-4, 0) {\rl{جناح سپاه}};
    \node[faction, fill=irangold!30] (esl) at (0, -2) {\rl{اصلاح‌طلبان}};
    
    \draw[<->, thick, dashed, red] (asl) -- (tanad) node[midway, above right, font=\tiny] {\rl{رقابت}};
    \draw[<->, thick, dashed, red] (sepah) -- (tanad) node[midway, below, font=\tiny] {\rl{تنش}};
    \draw[<->, thick, dashed, orange] (asl) -- (sepah) node[midway, above left, font=\tiny] {\rl{همکاری}};
\end{tikzpicture}
\captionof{figure}{\rl{تنش‌های جناحی درون نظام}}
\end{center}

\section{ظرفیت سرکوب: تحلیل واقع‌بینانه}

\begin{table}[H]
\centering
\caption{ماتریس ظرفیت سرکوب}
\begin{tabular}{|R{4.5cm}|C{2.5cm}|C{2.5cm}|C{2.5cm}|C{2.5cm}|}
\hline
\rowcolor{iranred!20}
\textbf{\rl{نوع سرکوب}} & \textbf{\rl{ظرفیت}} & \textbf{\rl{اراده}} & \textbf{\rl{هزینه}} & \textbf{\rl{پایداری}} \\
\hline
\rl{اعتراضات محدود} & \rl{بالا} & \rl{بالا} & \rl{پایین} & \rl{بالا} \\
\hline
\rl{خیزش گسترده} & \rl{متوسط} & \rl{بالا} & \rl{بالا} & \rl{متوسط} \\
\hline
\rl{اعتصاب سراسری} & \rl{پایین} & \rl{متوسط} & \rl{بالا} & \rl{پایین} \\
\hline
\rl{شورش مسلحانه} & \rl{بالا} & \rl{بالا} & \rl{بسیار بالا} & \rl{متوسط} \\
\hline
\end{tabular}
\end{table}

\section{جمع‌بندی: پتانسیل فروپاشی}

\begin{center}
\begin{tikzpicture}
    % SWOT-style matrix
    \draw[thick] (0,0) rectangle (12,8);
    \draw[thick] (6,0) -- (6,8);
    \draw[thick] (0,4) -- (12,4);
    
    % Headers
    \node[font=\large\bfseries, fill=olgoogreen!50, minimum width=5.8cm, minimum height=0.8cm] at (3, 7.6) {\rl{نقاط قوت}};
    \node[font=\large\bfseries, fill=iranred!30, minimum width=5.8cm, minimum height=0.8cm] at (9, 7.6) {\rl{نقاط ضعف}};
    \node[font=\large\bfseries, fill=iranblue!30, minimum width=5.8cm, minimum height=0.8cm] at (3, 3.6) {\rl{فرصت‌ها}};
    \node[font=\large\bfseries, fill=irangold!30, minimum width=5.8cm, minimum height=0.8cm] at (9, 3.6) {\rl{تهدیدها}};
    
    % Content (summarized)
    \node[text width=5cm, align=right, font=\small] at (3, 5.8) {\rl{دستگاه امنیتی قوی} / \rl{منابع مالی}};
    \node[text width=5cm, align=right, font=\small] at (9, 5.8) {\rl{بحران مشروعیت} / \rl{اقتصاد}};
    \node[text width=5cm, align=right, font=\small] at (3, 1.8) {\rl{فروپاشی از درون} / \rl{ریزش نیرو}};
    \node[text width=5cm, align=right, font=\small] at (9, 1.8) {\rl{سرکوب خونین} / \rl{جنگ داخلی}};
\end{tikzpicture}
\captionof{figure}{\rl{تحلیل SWOT نظام جمهوری اسلامی}}
\end{center}

\begin{kholasebox}[title=خلاصه فصل]
\begin{itemize}
    \item نظام جمهوری اسلامی ترکیبی از تئوکراسی، اقتدارگرایی و کلپتوکراسی است
    \item ستون‌های اصلی قدرت: نظامی-امنیتی، ایدئولوژیک و اقتصادی
    \item نظام با بحران‌های چندگانه مشروعیت، اقتصاد و شکاف نسلی مواجه است
    \item ظرفیت سرکوب همچنان بالاست اما پایدار نیست
    \item نقاط ضعف متعددی برای بهره‌برداری استراتژیک وجود دارد
\end{itemize}
\end{kholasebox}
