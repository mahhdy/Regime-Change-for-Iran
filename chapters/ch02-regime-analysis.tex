% ch02-regime-analysis.tex
% فصل دوم: تشریح نظام حاکم

\chapter{تشریح نظام حاکم}
\label{ch:regime-analysis}

\begin{center}
\begin{tikzpicture}[scale=0.9, transform shape]
    % Tower of Power
    \draw[thick, iranblue, fill=iranblue!10] (-3, 0) -- (3, 0) -- (1, 6) -- (-1, 6) -- cycle;
    
    % Core levels
    \node[rectangle, draw=iranred, fill=iranred!10, minimum width=1.5cm, minimum height=0.8cm] at (0, 5.2) {رهبری};
    \node[rectangle, draw=iranblue, fill=iranblue!20, minimum width=3cm, minimum height=0.8cm] at (0, 4) {سپاه و بسیج};
    \node[rectangle, draw=irangold, fill=irangold!20, minimum width=4.5cm, minimum height=0.8cm] at (0, 2.5) {اقتصاد و رانت};
    \node[rectangle, draw=irangreen, fill=irangreen!20, minimum width=5.5cm, minimum height=0.8cm] at (0, 1) {ایدئولوژی};
    
    % Pillars holding it up (symbolic)
    \draw[ultra thick, gray] (-2.5, -1) -- (-2.5, 0);
    \draw[ultra thick, gray] (2.5, -1) -- (2.5, 0);
    \draw[ultra thick, gray] (0, -1) -- (0, 0);
    
    \node[below=1.2cm of goal, align=center, font= \small \color{iranred!70}] at (0,-1) {ستون‌های بقای رژیم};
\end{tikzpicture}
\captionof{figure}{ساختار عمودی قدرت در رژیم}
\end{center}

\vspace{0.5cm}

\begin{naghlbox}
«برای شکست دادن دشمن، ابتدا باید او را شناخت. هر نقطه قوت، نقطه ضعفی پنهان دارد.»
\end{naghlbox}

\section{ماهیت نظام: شبه‌تمامیت‌خواه}

جمهوری اسلامی را نمی‌توان در دسته‌بندی‌های سنتی جای داد. این نظام ترکیبی از عناصر مختلف است:

\begin{center}
\begin{tikzpicture}[
    every node/.style={font=\small},
    box/.style={rectangle, rounded corners, draw, minimum width=3cm, minimum height=1cm, align=center}
]
    \node[box, fill=iranblue!20] (center) at (0,0) {جمهوری\\اسلامی};
    
    \node[box, fill=iranred!20] (theo) at (-4, 2) {تئوکراسی};
    \node[box, fill=irangold!20] (auth) at (0, 3) {اقتدارگرایی};
    \node[box, fill=empirepurple!20] (total) at (4, 2) {شبه‌تمامیت‌خواهی};
    \node[box, fill=olgoogreen!20] (klept) at (-4, -2) {کلپتوکراسی};
    \node[box, fill=noktegray] (preto) at (4, -2) {پرتوریانیسم\\(حکومت نظامی)};
    
    \draw[thick, ->] (center) -- (theo);
    \draw[thick, ->] (center) -- (auth);
    \draw[thick, ->] (center) -- (total);
    \draw[thick, ->] (center) -- (klept);
    \draw[thick, ->] (center) -- (preto);
\end{tikzpicture}
\captionof{figure}{ترکیب عناصر نظام جمهوری اسلامی}
\end{center}

\subsection{ویژگی‌های کلیدی}

\begin{table}[H]
\centering
\caption{مشخصات نظام جمهوری اسلامی}
\begin{tabular}{|R{3cm}|L{9cm}|}
\hline
\rowcolor{iranblue!20}
\textbf{ویژگی} & \textbf{توضیح} \\
\hline
ولایت فقیه & رهبر دارای قدرت نامحدود و غیرپاسخگو \\
\hline
دوگانگی قدرت & نهادهای موازی انتخابی و انتصابی \\
\hline
ایدئولوژی مذهبی & اسلام سیاسی شیعی \\
\hline
اقتصاد رانتی & وابستگی به نفت و کنترل دولتی \\
\hline
سرکوب نظام‌یافته & دستگاه امنیتی گسترده \\
\hline
\end{tabular}
\end{table}

\section{ستون‌های قدرت نظام}

\begin{empirebox}[title=چارچوب تحلیلی]
طبق نظریه جین شارپ، هر نظام استبدادی بر تعدادی «ستون قدرت» استوار است. تضعیف این ستون‌ها کلید سرنگونی نظام است.
\end{empirebox}

\subsection{ستون نظامی-امنیتی}

\begin{center}
\begin{tikzpicture}[
    org/.style={rectangle, rounded corners, draw=iranblue, fill=kholaseblue, minimum width=2.5cm, minimum height=0.8cm, font=\small, align=center},
    level 1/.style={sibling distance=4cm},
    level 2/.style={sibling distance=2.5cm},
    edge from parent/.style={draw, thick}
]
    \node[org, fill=iranblue!40] {رهبری}
        child { node[org] {سپاه پاسداران}
            child { node[org, font=\tiny] {نیروی\\قدس} }
            child { node[org, font=\tiny] {بسیج} }
        }
        child { node[org] {ارتش}
            child { node[org, font=\tiny] {نیروی\\زمینی} }
            child { node[org, font=\tiny] {نیروی\\هوایی} }
        }
        child { node[org] {اطلاعات}
            child { node[org, font=\tiny] {وزارت\\اطلاعات} }
            child { node[org, font=\tiny] {اطلاعات\\سپاه} }
        };
\end{tikzpicture}
\captionof{figure}{ساختار نیروهای نظامی-امنیتی}
\end{center}

\subsubsection{سپاه پاسداران انقلاب اسلامی}

\begin{itemize}
    \item \textbf{نفرات:} حدود ۱۹۰,۰۰۰ نفر (بدون بسیج)
    \item \textbf{بودجه:} تخمین ۲۰-۳۰ میلیارد دلار سالانه
    \item \textbf{امپراتوری اقتصادی:} کنترل بخش بزرگی از اقتصاد
    \item \textbf{وابستگی به نظام:} کامل - سرنوشت سپاه به بقای نظام گره خورده
\end{itemize}

\subsubsection{بسیج مستضعفین}

\begin{itemize}
    \item \textbf{عضویت رسمی:} ۲۰ میلیون نفر (ادعای رسمی)
    \item \textbf{نیروی فعال:} تخمین ۵۰۰,۰۰۰-۱,۰۰۰,۰۰۰ نفر
    \item \textbf{نقش:} سرکوب داخلی، جاسوسی اجتماعی
    \item \textbf{انگیزه:} ترکیبی از ایدئولوژی و منافع مادی
\end{itemize}

\subsection{ستون ایدئولوژیک-مذهبی}

\begin{table}[H]
\centering
\begin{tabular}{|R{3cm}|C{2cm}|L{7cm}|}
\hline
\rowcolor{irangold!30}
\textbf{نهاد} & \textbf{بودجه سالانه} & \textbf{نقش} \\
\hline
حوزه‌های علمیه & ۲ میلیارد دلار & تولید کادر مذهبی-سیاسی \\
\hline
صدا و سیما & ۱ میلیارد دلار & تبلیغات و کنترل اطلاعات \\
\hline
نهاد تبلیغات & ۵۰۰ میلیون دلار & گسترش ایدئولوژی \\
\hline
وزارت ارشاد & ۳۰۰ میلیون دلار & سانسور و کنترل فرهنگی \\
\hline
\end{tabular}
\caption{نهادهای ایدئولوژیک و بودجه تخمینی}
\end{table}

\subsection{ستون اقتصادی-رانتی}

\begin{center}
\begin{tikzpicture}
    \begin{axis}[
        ybar,
        ylabel={درصد از GDP},
        symbolic x coords={نفت و گاز, سپاه و بنیادها, بخش خصوصی, بخش دولتی},
        xtick=data,
        nodes near coords,
        nodes near coords align={vertical},
        ymin=0,
        ymax=50,
        bar width=25pt,
        x tick label style={rotate=45, anchor=east, font=\small},
        width=12cm,
        height=8cm
    ]
    \addplot[fill=iranblue!50] coordinates {
        (نفت و گاز, 40)
        (سپاه و بنیادها, 25)
        (بخش خصوصی, 20)
        (بخش دولتی, 15)
    };
    \end{axis}
\end{tikzpicture}
\captionof{figure}{توزیع تخمینی اقتصاد ایران}
\end{center}

\subsection{شبکه‌های وابستگی منطقه‌ای}

\begin{table}[H]
\centering
\caption{نیروهای نیابتی جمهوری اسلامی}
\begin{tabular}{|R{3cm}|C{2.5cm}|C{3cm}|L{3.5cm}|}
\hline
\rowcolor{iranred!20}
\textbf{گروه} & \textbf{کشور} & \textbf{نفرات تخمینی} & \textbf{وضعیت فعلی} \\
\hline
حزب‌الله لبنان & لبنان & ۴۵,۰۰۰ & تضعیف شدید \\
\hline
حشد الشعبی & عراق & ۱۵۰,۰۰۰ & تحت فشار \\
\hline
حوثی‌ها & یمن & ۲۰۰,۰۰۰ & فعال \\
\hline
گروه‌های فلسطینی & غزه/... & متغیر & تضعیف شده \\
\hline
\end{tabular}
\end{table}

\section{نقاط ضعف و شکاف‌ها}

\begin{enghelabbox}[title=نقطه کلیدی]
هر نظامی، هرقدر هم قدرتمند، دارای نقاط ضعف است. شناسایی و بهره‌برداری از این نقاط، اساس استراتژی گذار است.
\end{enghelabbox}

\subsection{بحران مشروعیت}

\begin{center}
\begin{tikzpicture}
    \begin{axis}[
        xlabel={سال},
        ylabel={درصد رضایت (تخمینی)},
        xmin=1378, xmax=1404,
        ymin=0, ymax=100,
        xtick={1378, 1388, 1396, 1401, 1404},
        xticklabels={۱۳۷۸, ۱۳۸۸, ۱۳۹۶, ۱۴۰۱, ۱۴۰۴},
        width=12cm,
        height=7cm,
        grid=major,
        legend pos=north east
    ]
    \addplot[thick, color=iranblue, mark=*] coordinates {
        (1378, 65)
        (1388, 45)
        (1396, 30)
        (1401, 15)
        (1404, 8)
    };
    \legend{رضایت از نظام}
    \end{axis}
\end{tikzpicture}
\captionof{figure}{روند نزولی مشروعیت نظام (تخمینی)}
\end{center}

\subsection{بحران اقتصادی}

\begin{itemize}
    \item \textbf{تورم:} بیش از ۵۰٪ سالانه
    \item \textbf{بیکاری:} رسمی ۱۲٪، واقعی ۲۵-۳۰٪
    \item \textbf{فقر:} بیش از ۳۰٪ زیر خط فقر
    \item \textbf{ارزش پول ملی:} سقوط ۹۰٪ در دهه گذشته
\end{itemize}

\subsection{شکاف نسلی}

\begin{noktebox}
بیش از ۶۰٪ جمعیت ایران زیر ۴۰ سال سن دارند. این نسل نه انقلاب ۵۷ را به یاد دارد و نه تعلقی به آرمان‌های نظام.
\end{noktebox}

\subsection{شکاف مرکز-پیرامون}

\begin{table}[H]
\centering
\begin{tabular}{|R{3cm}|C{2.5cm}|L{6cm}|}
\hline
\rowcolor{olgoogreen!30}
\textbf{منطقه} & \textbf{جمعیت تخمینی} & \textbf{موضوع اصلی} \\
\hline
کردستان & ۱۰ میلیون & خودمختاری، سرکوب فرهنگی \\
\hline
بلوچستان & ۳ میلیون & فقر، تبعیض، سرکوب \\
\hline
خوزستان & ۵ میلیون & آب، محیط زیست، تبعیض \\
\hline
آذربایجان & ۱۵ میلیون & حقوق زبانی، توسعه \\
\hline
\end{tabular}
\caption{شکاف‌های منطقه‌ای}
\end{table}

\subsection{تنش‌های درون‌حکومتی}

\begin{center}
\begin{tikzpicture}[
    faction/.style={ellipse, draw, minimum width=3cm, minimum height=1.5cm, align=center, font=\small}
]
    \node[faction, fill=iranblue!30] (asl) at (0, 2) {اصول‌گرایان\\سنتی};
    \node[faction, fill=iranred!30] (tanad) at (4, 0) {تندروها\\(پایداری)};
    \node[faction, fill=olgoogreen!30] (sepah) at (-4, 0) {جناح\\سپاه};
    \node[faction, fill=irangold!30] (esl) at (0, -2) {اصلاح‌طلبان\\(حاشیه‌ای)};
    
    \draw[<->, thick, dashed, red] (asl) -- (tanad) node[midway, above right, font=\tiny] {رقابت};
    \draw[<->, thick, dashed, red] (sepah) -- (tanad) node[midway, below, font=\tiny] {تنش};
    \draw[<->, thick, dashed, orange] (asl) -- (sepah) node[midway, above left, font=\tiny] {همکاری محتاطانه};
\end{tikzpicture}
\captionof{figure}{تنش‌های جناحی درون نظام}
\end{center}

\section{ظرفیت سرکوب: تحلیل واقع‌بینانه}

\begin{table}[H]
\centering
\caption{ماتریس ظرفیت سرکوب}
\begin{tabular}{|R{3.5cm}|C{2cm}|C{2cm}|C{2cm}|C{2cm}|}
\hline
\rowcolor{iranred!20}
\textbf{نوع سرکوب} & \textbf{ظرفیت} & \textbf{اراده} & \textbf{هزینه} & \textbf{پایداری} \\
\hline
اعتراضات محدود & بالا & بالا & پایین & بالا \\
\hline
خیزش گسترده & متوسط & بالا & بالا & متوسط \\
\hline
اعتصاب سراسری & پایین & متوسط & بالا & پایین \\
\hline
شورش مسلحانه & بالا & بالا & بسیار بالا & متوسط \\
\hline
\end{tabular}
\end{table}

\section{جمع‌بندی: پتانسیل فروپاشی}

\begin{center}
\begin{tikzpicture}
    % SWOT-style matrix
    \draw[thick] (0,0) rectangle (12,8);
    \draw[thick] (6,0) -- (6,8);
    \draw[thick] (0,4) -- (12,4);
    
    % Headers
    \node[font=\large\bfseries, fill=olgoogreen!50, minimum width=5.8cm, minimum height=0.8cm] at (3, 7.6) {نقاط قوت};
    \node[font=\large\bfseries, fill=iranred!30, minimum width=5.8cm, minimum height=0.8cm] at (9, 7.6) {نقاط ضعف};
    \node[font=\large\bfseries, fill=iranblue!30, minimum width=5.8cm, minimum height=0.8cm] at (3, 3.6) {فرصت‌ها};
    \node[font=\large\bfseries, fill=irangold!30, minimum width=5.8cm, minimum height=0.8cm] at (9, 3.6) {تهدیدها};
    
    % Content
    \node[text width=5cm, align=right, font=\small] at (3, 5.8) {
        • دستگاه امنیتی قوی\\
        • انسجام نسبی رهبری\\
        • منابع مالی (نفت)\\
        • نیروهای نیابتی
    };
    
    \node[text width=5cm, align=right, font=\small] at (9, 5.8) {
        • بحران مشروعیت\\
        • بحران اقتصادی\\
        • شکاف نسلی\\
        • انزوای بین‌المللی
    };
    
    \node[text width=5cm, align=right, font=\small] at (3, 1.8) {
        • فروپاشی از درون\\
        • ریزش نیروها\\
        • تحولات منطقه‌ای\\
        • فشار بین‌المللی
    };
    
    \node[text width=5cm, align=right, font=\small] at (9, 1.8) {
        • سرکوب خونین\\
        • جنگ داخلی\\
        • مداخله خارجی منفی\\
        • تجزیه کشور
    };
\end{tikzpicture}
\captionof{figure}{تحلیل SWOT نظام جمهوری اسلامی}
\end{center}

\begin{kholasebox}[title=خلاصه فصل]
\begin{itemize}
    \item نظام جمهوری اسلامی ترکیبی از تئوکراسی، اقتدارگرایی و کلپتوکراسی است
    \item ستون‌های اصلی قدرت: نظامی-امنیتی، ایدئولوژیک و اقتصادی
    \item نظام با بحران‌های چندگانه مشروعیت، اقتصاد و شکاف نسلی مواجه است
    \item ظرفیت سرکوب همچنان بالاست اما پایدار نیست
    \item نقاط ضعف متعددی برای بهره‌برداری استراتژیک وجود دارد
\end{itemize}
\end{kholasebox}
