% ch05-international.tex
% فصل پنجم: بُعد بین‌المللی

\chapter{بُعد بین‌المللی}
\label{ch:international}

\begin{center}
\begin{tikzpicture}[scale=0.9, transform shape]
    % World Map (Symbolic)
    \draw[gray!20, fill=noktegray] (-5, -3) rectangle (5, 3);
    
    % Actors
    \node[concept, scale=0.6, fill=iranblue!20] (usa) at (-3, 1) {\rl{آمریکا}};
    \node[concept, scale=0.6, fill=iranblue!15] (eu) at (-1, 1.5) {\rl{اروپا}};
    \node[concept, scale=0.6, fill=iranred!15] (rus) at (2, 2) {\rl{روسیه}};
    \node[concept, scale=0.6, fill=iranred!20] (chn) at (4, 1.5) {\rl{چین}};
    \node[concept, scale=0.6, fill=irangold!20] (sa) at (1, -1) {\rl{عربستان}};
    \node[concept, scale=0.6, fill=iranred!10] (isr) at (-0.5, -0.5) {\rl{اسرائیل}};
    
    % Center: Iran
    \node[concept, scale=0.8, fill=white, draw=iranblue] (iran) at (0, 0.5) {\rl{ایران}};
    
    % Lines (Support/Pressure)
    \draw[connection, iranred] (rus) -- (iran);
    \draw[connection, iranred] (chn) -- (iran);
    \draw[connection, iranblue] (usa) -- (iran);
    \draw[connection, iranblue] (eu) -- (iran);
    \draw[connection, irangold] (sa) -- (iran);
    \draw[connection, iranblue] (isr) -- (iran);
    
    \node[below=3.5cm of goal, align=center, font= \small \color{iranblue!70}] at (0,0) {\rl{تأثیرگذاری بازیگران بر وضعیت ایران}};
\end{tikzpicture}
\captionof{figure}{\rl{نقشه بازیگران بین‌المللی و فشارهای دیپلماتیک}}
\end{center}

\begin{naghlbox}
«درک منافع بازیگران بین‌المللی کلید بهره‌برداری از آنهاست.»
\end{naghlbox}

\section{بازیگران کلیدی}

\begin{table}[H]
\centering
\caption{تحلیل بازیگران اصلی}
\begin{tabular}{|R{3.5cm}|C{3cm}|R{9cm}|}
\hline
\rowcolor{iranblue!20}
\textbf{\rl{بازیگر}} & \textbf{\rl{موضع}} & \textbf{\rl{منافع کلیدی}} \\
\hline
\rl{آمریکا} & \rl{ضد نظام} & \rl{جلوگیری از برنامه هسته‌ای، تضعیف نفوذ منطقه‌ای} \\
\hline
\rl{اسرائیل} & \rl{ضد نظام} & \rl{خنثی‌سازی تهدید، تضعیف نیروهای نیابتی} \\
\hline
\rl{اتحادیه اروپا} & \rl{محتاط} & \rl{ثبات منطقه‌ای، تجارت، حقوق بشر} \\
\hline
\rl{روسیه} & \rl{حامی نظام} & \rl{همکاری نظامی، مقابله با غرب} \\
\hline
\rl{چین} & \rl{حامی نظام} & \rl{نفت، جاده ابریشم، موازنه قدرت} \\
\hline
\rl{عربستان} & \rl{ضد نظام} & \rl{رقابت منطقه‌ای، امنیت} \\
\hline
\end{tabular}
\end{table}

\section{انواع مداخله}

\subsection{تحریم‌ها و فشار اقتصادی}
\begin{itemize}
    \item تحریم نفتی: کاهش ۸۰٪ صادرات
    \item تحریم بانکی: قطع دسترسی به سیستم مالی
    \item تحریم افراد: مسدود کردن دارایی‌های مقامات
\end{itemize}

\subsection{حمایت دیپلماتیک}
\begin{olgoobox}
\begin{itemize}
    \item شناسایی حکومت انتقالی
    \item محکومیت‌های بین‌المللی
    \item حمایت از محاکمه در ICC
\end{itemize}
\end{olgoobox}

\subsection{کمک غیرنظامی}
رسانه، فناوری دور زدن فیلترینگ، آموزش کادرها، کمک مالی.

\subsection{مداخله نظامی}
\begin{enghelabbox}[title=هشدار]
مداخله نظامی گسترده غیرمحتمل و مخرب است. درس‌های عراق و افغانستان.
\end{enghelabbox}

\section{فرصت‌های کنونی}

\begin{empirebox}
\begin{enumerate}
    \item تضعیف نیروهای نیابتی
    \item انزوای بیشتر نظام
    \item توجه بین‌المللی پس از رویدادهای اخیر
    \item اجماع نسبی غرب و اعراب
\end{enumerate}
\end{empirebox}

\section{توصیه‌ها برای دیپلماسی اپوزیسیون}

\begin{table}[H]
\centering
\begin{tabular}{|R{4cm}|R{11cm}|}
\hline
\rowcolor{olgoogreen!30}
\textbf{\rl{توصیه}} & \textbf{\rl{توضیح}} \\
\hline
\rl{وحدت در پیام} & \rl{یک صدای واحد در محافل بین‌المللی} \\
\hline
\rl{استقلال} & \rl{اجتناب از وابستگی به هر قدرت خارجی} \\
\hline
\rl{لابی مؤثر} & \rl{حضور در پارلمان‌ها و نهادهای بین‌المللی} \\
\hline
\rl{بسیج دیاسپورا} & \rl{فشار ایرانیان خارج بر دولت‌ها} \\
\hline
\end{tabular}
\end{table}

\begin{kholasebox}[title=خلاصه فصل]
\begin{itemize}
    \item بازیگران بین‌المللی منافع متفاوتی دارند
    \item تحریم‌ها مؤثرترین ابزار فعلی هستند
    \item مداخله نظامی گسترده غیرمحتمل است
    \item اپوزیسیون باید مستقل اما عمل‌گرا باشد
\end{itemize}
\end{kholasebox}
