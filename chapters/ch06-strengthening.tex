% ch06-strengthening.tex
% فصل ششم: تقویت اپوزیسیون

\chapter{تقویت اپوزیسیون}
\label{ch:strengthening}

\begin{center}
\begin{tikzpicture}[scale=0.9, transform shape]
    % Strengthening the fist (symbolic)
    \node[concept, scale=1.2, fill=iranblue!20] (fist) at (0,0) {اتحاد و\\قدرت};
    
    % Layers of strengthening
    \node[step] (s1) at (-4, 2) {آموزش و تخصص};
    \node[step] (s2) at (4, 2) {منابع و لجستیک};
    \node[step] (s3) at (-4, -2) {رسانه و تبلیغات};
    \node[step] (s4) at (4, -2) {امنیت و حفاظت};
    
    \draw[connection] (s1) -- (fist);
    \draw[connection] (s2) -- (fist);
    \draw[connection] (s3) -- (fist);
    \draw[connection] (s4) -- (fist);
    
    \node[below=2.5cm of fist, align=center, font= \small \color{iranblue!70}] {ارکان تقویت نیروهای اپوزیسیون};
\end{tikzpicture}
\captionof{figure}{مدل ارتقای توانمندی‌های مبارزاتی}
\end{center}

\begin{naghlbox}
«پیروزی از آنِ کسی است که هم خود را قوی کند و هم دشمن را ضعیف.»
\end{naghlbox}

\section{چگونه ما قوی‌تر شویم؟}

\subsection{ائتلاف‌سازی عملی}

\begin{empirebox}[title=اصول ائتلاف موفق]
\begin{enumerate}
    \item \textbf{توافق بر حداقل‌ها:} سرنگونی، دموکراسی، حقوق بشر
    \item \textbf{پذیرش تنوع:} اختلاف نظر طبیعی و سالم است
    \item \textbf{ساختار شفاف:} مسئولیت‌ها و تصمیم‌گیری روشن
    \item \textbf{مکانیزم حل اختلاف:} رویه‌های مشخص
\end{enumerate}
\end{empirebox}

\begin{center}
\begin{tikzpicture}[node distance=2cm]
    \node[block, fill=iranblue!30] (min) {توافق حداقلی};
    \node[block, fill=iranblue!40, right=of min] (struct) {ساختارسازی};
    \node[block, fill=iranblue!50, right=of struct] (coord) {هماهنگی عملی};
    \node[block, fill=olgoogreen!50, right=of coord] (action) {اقدام مشترک};
    
    \draw[arrow] (min) -- (struct);
    \draw[arrow] (struct) -- (coord);
    \draw[arrow] (coord) -- (action);
\end{tikzpicture}
\captionof{figure}{مسیر ائتلاف‌سازی}
\end{center}

\subsection{ساختارسازی سازمانی}

\begin{table}[H]
\centering
\begin{tabular}{|R{3cm}|L{9cm}|}
\hline
\rowcolor{iranblue!20}
\textbf{نهاد پیشنهادی} & \textbf{وظیفه} \\
\hline
شورای هماهنگی & تصمیم‌گیری کلان، نمایندگی بین‌المللی \\
\hline
کمیته رسانه & هماهنگی پیام‌رسانی، مبارزه با تحریف \\
\hline
کمیته حقوقی & مستندسازی، عدالت انتقالی \\
\hline
کمیته ارتباط با داخل & شبکه‌سازی امن، انتقال اطلاعات \\
\hline
کمیته منابع & جذب و مدیریت شفاف منابع مالی \\
\hline
\end{tabular}
\caption{ساختار پیشنهادی اپوزیسیون متحد}
\end{table}

\subsection{ارتباط با داخل کشور}

\begin{enghelabbox}[title=چالش امنیتی]
ارتباط با داخل پرخطر است. باید با رعایت کامل اصول امنیتی صورت گیرد.
\end{enghelabbox}

\begin{itemize}
    \item استفاده از ابزارهای رمزنگاری‌شده
    \item شبکه‌های غیرمتمرکز و سلولی
    \item آموزش امنیت دیجیتال
    \item اجتناب از افشای هویت فعالان
\end{itemize}

\section{چگونه حکومت ضعیف‌تر شود؟}

\subsection{شکستن ستون‌های قدرت}

\begin{center}
\begin{tikzpicture}
    % Pillars
    \draw[fill=iranred!30] (0,0) rectangle (2,4);
    \draw[fill=iranred!30] (3,0) rectangle (5,4);
    \draw[fill=iranred!30] (6,0) rectangle (8,4);
    \draw[fill=iranred!30] (9,0) rectangle (11,4);
    
    % Labels
    \node[font=\small, rotate=90] at (1, 2) {نظامی-امنیتی};
    \node[font=\small, rotate=90] at (4, 2) {ایدئولوژیک};
    \node[font=\small, rotate=90] at (7, 2) {اقتصادی};
    \node[font=\small, rotate=90] at (10, 2) {بین‌المللی};
    
    % Cracks
    \draw[thick, white, decorate, decoration={zigzag, segment length=4mm, amplitude=2mm}] (1, 0.5) -- (1, 3.5);
    \draw[thick, white, decorate, decoration={zigzag, segment length=4mm, amplitude=2mm}] (4, 0.5) -- (4, 3.5);
    
    % Roof
    \draw[fill=iranblue!30] (-0.5, 4) -- (5.5, 5) -- (11.5, 4) -- cycle;
    \node at (5.5, 4.5) {نظام};
\end{tikzpicture}
\captionof{figure}{ستون‌های قدرت نظام}
\end{center}

\subsection{نافرمانی مدنی}

\begin{olgoobox}[title=اشکال نافرمانی مدنی]
\begin{enumerate}
    \item اعتصاب عمومی و صنفی
    \item تحریم انتخابات و نهادها
    \item عدم پرداخت قبوض دولتی
    \item اعتراضات خیابانی
    \item کمپین‌های رسانه‌ای
\end{enumerate}
\end{olgoobox}

\section{استراتژی‌های ارتباطی و رسانه‌ای}

\begin{table}[H]
\centering
\begin{tabular}{|R{2.5cm}|L{5cm}|L{4.5cm}|}
\hline
\rowcolor{irangold!20}
\textbf{پلتفرم} & \textbf{کاربرد} & \textbf{چالش} \\
\hline
تلویزیون ماهواره‌ای & دسترسی گسترده، اخبار & هزینه بالا \\
\hline
شبکه‌های اجتماعی & بسیج سریع، ویدئو & فیلترینگ \\
\hline
پیام‌رسان‌ها & ارتباط مستقیم & نظارت امنیتی \\
\hline
\end{tabular}
\caption{ابزارهای ارتباطی}
\end{table}

\section{نقش دیاسپورا}

\begin{itemize}
    \item \textbf{لابی‌گری:} فشار بر دولت‌ها و نمایندگان
    \item \textbf{تأمین مالی:} کمک به فعالیت‌های اپوزیسیون
    \item \textbf{تخصص:} ارائه دانش فنی و مدیریتی
    \item \textbf{رسانه:} تولید و انتشار محتوا
\end{itemize}

\section{اولویت‌بندی اقدامات}

\begin{table}[H]
\centering
\begin{tabular}{|C{1cm}|L{6cm}|C{2cm}|C{2cm}|}
\hline
\rowcolor{olgoogreen!30}
\textbf{\#} & \textbf{اقدام} & \textbf{فوریت} & \textbf{تأثیر} \\
\hline
۱ & ائتلاف‌سازی & بالا & بالا \\
\hline
۲ & ساختارسازی & بالا & بالا \\
\hline
۳ & ارتباط با داخل & بالا & متوسط \\
\hline
۴ & دیپلماسی & متوسط & بالا \\
\hline
۵ & جذب منابع & متوسط & متوسط \\
\hline
\end{tabular}
\caption{ماتریس اولویت‌بندی}
\end{table}

\begin{kholasebox}[title=خلاصه فصل]
\begin{itemize}
    \item ائتلاف بر پایه توافق حداقلی ممکن است
    \item ساختارسازی سازمانی ضروری است
    \item ارتباط امن با داخل حیاتی است
    \item نافرمانی مدنی ستون‌های قدرت را تضعیف می‌کند
    \item دیاسپورا نقش کلیدی دارد
\end{itemize}
\end{kholasebox}
