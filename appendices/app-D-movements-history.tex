% app-D-movements-history.tex
% پیوست د: تاریخ جنبش‌های ایران

\chapter{تاریخ جنبش‌های ایران}
\label{app:movements-history}

\begin{center}
\begin{tikzpicture}[scale=0.9, transform shape]
    \node[concept] (hist) at (0,0) {\rl{تاریخ مبارزه}};
    \node[step] (s1) at (-4, 2) {\rl{مشروطه و گذشته}};
    \node[step] (s2) at (4, 2) {\rl{انقلاب ۵۷}};
    \node[step] (s3) at (0, -3) {\rl{خیزش‌های نوین}};
    
    \draw[connection] (hist) -- (s1);
    \draw[connection] (hist) -- (s2);
    \draw[connection] (hist) -- (s3);
\end{tikzpicture}
\captionof{figure}{\rl{سیر تکامل جنبش‌های اعتراضی در ایران}}
\end{center}

\section{\rl{خط زمانی جنبش‌ها}}

\begin{center}
\begin{tikzpicture}[scale=0.9]
    % Timeline
    \draw[thick, iranblue] (0,0) -- (14,0);
    
    % Events
    \foreach \x/\year/\event/\col in {
        0/۵۷/\rl{انقلاب}/iranred,
        2/۷۸/\rl{۱۸ تیر}/iranblue,
        4/۸۸/\rl{جنبش سبز}/olgoogreen,
        7/۹۶/\rl{دی ۹۶}/irangold,
        9/۹۸/\rl{آبان ۹۸}/iranred,
        11/۱۴۰۱/\rl{مهسا}/empirepurple,
        13/۱۴۰۴/\rl{دی ۰۴}/iranred
    }{
        \draw[thick, \col] (\x, -0.3) -- (\x, 0.3);
        \fill[\col] (\x, 0) circle (0.15);
        \node[above, font=\small\bfseries] at (\x, 0.4) {\year};
        \node[below, font=\tiny, text width=1.2cm, align=center] at (\x, -0.5) {\event};
    }
\end{tikzpicture}
\captionof{figure}{\rl{خط زمانی جنبش‌های اعتراضی}}
\end{center}

\section{جنبش دانشجویی ۱۸ تیر ۱۳۷۸}

\begin{itemize}
    \item \textbf{علت:} بسته شدن روزنامه سلام
    \item \textbf{گستره:} دانشگاه تهران، سپس سایر شهرها
    \item \textbf{سرکوب:} حمله انصار حزب‌الله
    \item \textbf{درس:} اهمیت گسترش به خارج از دانشگاه
\end{itemize}

\section{جنبش سبز ۱۳۸۸}

\begin{table}[H]
\centering
\begin{tabular}{|R{4cm}|R{11cm}|}
\hline
\rowcolor{olgoogreen!30}
\textbf{\rl{ویژگی}} & \textbf{\rl{توضیح}} \\
\hline
\rl{علت} & \rl{تقلب انتخاباتی} \\
\hline
\rl{رهبری} & \rl{موسوی و کروبی} \\
\hline
\rl{شعار} & \rl{«رأی من کو؟»} \\
\hline
\rl{گستره} & \rl{میلیون‌ها نفر در تهران} \\
\hline
\rl{سرکوب} & \rl{کهریزک، کشتار، زندان} \\
\hline
\rl{نتیجه} & \rl{شکست، حصر رهبران} \\
\hline
\end{tabular}
\caption{\rl{مشخصات جنبش سبز}}
\end{table}

\section{خیزش دی‌ماه ۱۳۹۶}

اولین جنبش ضدنظام (نه اصلاح‌طلبانه):
\begin{itemize}
    \item شعار «اصلاح‌طلب، اصول‌گرا، دیگه تمومه ماجرا»
    \item گستره: شهرستان‌ها، طبقات پایین
    \item سرکوب سریع
\end{itemize}

\section{خیزش آبان ۱۳۹۸}

\begin{enghelabbox}[title=خونین‌ترین سرکوب]
\begin{itemize}
    \item علت: افزایش قیمت بنزین
    \item کشته: ۱۵۰۰+ (طبق رویترز)
    \item قطع اینترنت: ۱ هفته
    \item گستره: سراسری
\end{itemize}
\end{enghelabbox}

\section{جنبش مهسا ۱۴۰۱}

\begin{itemize}
    \item علت: کشته شدن مهسا امینی توسط گشت ارشاد
    \item شعار: «زن، زندگی، آزادی»
    \item ویژگی: رهبری زنان، نسل Z
    \item گستره: سراسری، همه اقشار
    \item مدت: چندین ماه
\end{itemize}

\section{روند تحول جنبش‌ها}

\begin{olgoobox}
\begin{enumerate}
    \item \textbf{از اصلاح به براندازی:} شعارها رادیکال‌تر شده
    \item \textbf{از نخبگان به توده:} شهرستان‌ها و طبقات پایین
    \item \textbf{از مردان به زنان:} نقش پیشتاز زنان
    \item \textbf{از ترس به شجاعت:} کاهش ترس از نظام
\end{enumerate}
\end{olgoobox}
