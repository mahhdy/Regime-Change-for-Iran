% app-F-human-rights.tex
% پیوست و: مستندات حقوق بشری

\chapter{مستندات حقوق بشری}
\label{app:human-rights}

\begin{center}
\begin{tikzpicture}[scale=0.9, transform shape]
    \node[concept, fill=iranred!10, draw=iranred] (hr) at (0,0) {\rl{حقوق بشر}};
    \node[step, draw=iranred!50] (s1) at (-4, 2) {\rl{اعدام و شکنجه}};
    \node[step, draw=iranred!50] (s2) at (4, 2) {\rl{سرکوب اقلیت‌ها}};
    \node[step, draw=iranred!50] (s3) at (0, -3) {\rl{تبعیض جنسیتی}};
    
    \draw[connection, iranred!30] (hr) -- (s1);
    \draw[connection, iranred!30] (hr) -- (s2);
    \draw[connection, iranred!30] (hr) -- (s3);
\end{tikzpicture}
\captionof{figure}{\rl{تارنمای نقض حقوق بشر در ایران}}
\end{center}

\section{اعدام‌ها}

\begin{table}[H]
\centering
\begin{tabular}{|C{3cm}|C{3cm}|R{9cm}|}
\hline
\rowcolor{iranred!20}
\textbf{\rl{سال}} & \textbf{\rl{تعداد اعدام}} & \textbf{\rl{توضیح}} \\
\hline
\rl{۱۴۰۱} & \rl{۵۸۲} & \rl{افزایش پس از جنبش مهسا} \\
\hline
\rl{۱۴۰۲} & \rl{۸۵۳} & \rl{بالاترین رقم در دهه اخیر} \\
\hline
\rl{۱۴۰۳} & \rl{۶۰۰+} & \rl{ادامه روند} \\
\hline
\end{tabular}
\caption{آمار اعدام سالانه (منبع: ایران هیومن رایتس)}
\end{table}

\section{زندانیان سیاسی}

\begin{itemize}
    \item تعداد تخمینی: ۲۰,۰۰۰+ نفر
    \item زنان: افزایش چشمگیر پس از ۱۴۰۱
    \item روزنامه‌نگاران: ۵۰+ نفر
    \item وکلا: ۲۰+ نفر
    \item فعالان کارگری: ۱۰۰+ نفر
\end{itemize}

\section{شکنجه و بدرفتاری}

\begin{enghelabbox}
گزارش‌های متعدد از:
\begin{itemize}
    \item شکنجه جسمی و روانی
    \item انفرادی طولانی‌مدت
    \item تجاوز جنسی
    \item محرومیت از درمان
    \item اعترافات اجباری
\end{itemize}
\end{enghelabbox}

\section{سرکوب زنان}

\begin{itemize}
    \item حجاب اجباری و گشت ارشاد
    \item محدودیت در آموزش و اشتغال
    \item قوانین تبعیض‌آمیز خانواده
    \item سرکوب معترضان زن
\end{itemize}

\section{سرکوب اقلیت‌ها}

\begin{table}[H]
\centering
\begin{tabular}{|R{4cm}|R{11cm}|}
\hline
\rowcolor{empirepurple!20}
\textbf{\rl{گروه}} & \textbf{\rl{نوع سرکوب}} \\
\hline
\rl{بهائیان} & \rl{محرومیت از تحصیل، زندان، مصادره اموال} \\
\hline
\rl{سنی‌ها} & \rl{ممنوعیت مسجد در تهران، تبعیض} \\
\hline
\rl{دراویش} & \rl{سرکوب خشونت‌آمیز، زندان} \\
\hline
\rl{کُردها} & \rl{اعدام، زندان، سرکوب فرهنگی} \\
\hline
\rl{بلوچ‌ها} & \rl{اعدام‌های گسترده، تبعیض} \\
\hline
\end{tabular}
\caption{سرکوب اقلیت‌ها}
\end{table}

\section{منابع مستندسازی}

\begin{olgoobox}
\begin{itemize}
    \item ایران هیومن رایتس (IHR)
    \item عفو بین‌الملل
    \item دیده‌بان حقوق بشر
    \item گزارشگر ویژه سازمان ملل
    \item HRANA
    \item کمیته گزارشگران بدون مرز
\end{itemize}
\end{olgoobox}

\section{اهمیت مستندسازی}

\begin{empirebox}
مستندسازی جنایات برای:
\begin{enumerate}
    \item فشار بین‌المللی
    \item عدالت انتقالی در آینده
    \item ثبت تاریخ
    \item حمایت از قربانیان
\end{enumerate}
\end{empirebox}
