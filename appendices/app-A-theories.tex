% app-A-theories.tex
% پیوست الف: نظریه‌های انقلاب و گذار

\chapter{نظریه‌های انقلاب و گذار}
\label{app:theories}

\begin{center}
\begin{tikzpicture}[scale=0.9, transform shape]
    \node[concept] (base) at (0,0) {\rl{نظریات گذار}};
    \node[step] (s1) at (-4, 2) {\rl{ساختارگرایی}};
    \node[step] (s2) at (4, 2) {\rl{بسیج منابع}};
    \node[step] (s3) at (0, -3) {\rl{گذارهای مذاکره‌ای}};
    
    \draw[connection] (base) -- (s1);
    \draw[connection] (base) -- (s2);
    \draw[connection] (base) -- (s3);
    
    \node[below=3.5cm of base, align=center, font=\small\color{iranblue!70}] {\rl{چارچوب‌های تئوریک مطالعه گذار سیاسی}};
\end{tikzpicture}
\captionof{figure}{\rl{تلفیق نظریه‌های کلیدی در مدل گذار ایران}}
\end{center}

\section{\rl{نظریه ساختاری اسکاچپول}}

\rl{تدا اسکاچپول در کتاب «دولت‌ها و انقلاب‌های اجتماعی» (۱۹۷۹) استدلال می‌کند که انقلاب‌ها زمانی رخ می‌دهند که:}

\begin{itemize}
    \item \rl{دولت دچار بحران مالی و نظامی شود}
    \item \rl{نخبگان از حکومت فاصله بگیرند}
    \item \rl{دهقانان/توده‌ها ظرفیت سازماندهی داشته باشند}
\end{itemize}

\begin{noktebox}
\textbf{\rl{کاربرد برای ایران:}} \rl{بحران مالی و فشار بین‌المللی موجود است. شکاف نخبگان رو به افزایش. ظرفیت سازماندهی توده‌ها (به ویژه طبقه متوسط شهری) نسبتاً بالا.}
\end{noktebox}

\section{\rl{نظریه بسیج منابع تیلی}}

\rl{چارلز تیلی تأکید می‌کند که جنبش‌های اجتماعی نیازمند:}
\begin{enumerate}
    \item \rl{منابع (پول، نیرو، سازمان)}
    \item \rl{فرصت سیاسی}
    \item \rl{چارچوب‌بندی مناسب (Framing)}
\end{enumerate}

\section{\rl{۱۹۸ روش جین شارپ}}

\rl{جین شارپ در «از دیکتاتوری به دموکراسی» ۱۹۸ روش مبارزه غیرخشونت‌آمیز را فهرست کرده:}

\begin{table}[H]
\centering
\begin{tabular}{|R{6cm}|C{3cm}|R{7cm}|}
\hline
\rowcolor{olgoogreen!30}
\textbf{\rl{دسته}} & \textbf{\rl{تعداد روش‌ها}} & \textbf{\rl{نمونه}} \\
\hline
\rl{اعتراض نمادین} & \rl{۵۴} & \rl{تظاهرات، امضا، نمادها} \\
\hline
\rl{عدم همکاری اجتماعی} & \rl{۱۶} & \rl{اعتصاب اجتماعی} \\
\hline
\rl{عدم همکاری اقتصادی} & \rl{۴۹} & \rl{اعتصاب، تحریم خرید} \\
\hline
\rl{عدم همکاری سیاسی} & \rl{۳۸} & \rl{تحریم انتخابات، نافرمانی} \\
\hline
\rl{مداخله غیرخشونت‌آمیز} & \rl{۴۱} & \rl{تحصن، اشغال} \\
\hline
\end{tabular}
\caption{\rl{دسته‌بندی روش‌های جین شارپ}}
\end{table}

\section{\rl{انقلاب‌های مخملی: الزامات}}

\rl{تجربه اروپای شرقی نشان می‌دهد انقلاب موفق غیرخشونت‌آمیز نیازمند:}

\begin{olgoobox}
\begin{enumerate}
    \item \rl{رهبری متحد و شناخته‌شده}
    \item \rl{سازمان‌های مدنی قوی}
    \item \rl{رسانه‌های مستقل}
    \item \rl{حمایت بین‌المللی}
    \item \rl{شکاف در نیروهای امنیتی}
    \item \rl{یک رویداد آغازگر}
\end{enumerate}
\end{olgoobox}

\section{\rl{نظریه گذار دموکراتیک}}

\rl{اودانل و اشمیتر چهار مرحله گذار را تعریف کردند:}

\begin{center}
\begin{tikzpicture}[node distance=2cm]
    \node[block, fill=iranred!30] (auth) {\rl{اقتدارگرایی}};
    \node[block, fill=irangold!30, right=of auth] (lib) {\rl{لیبرالیزاسیون}};
    \node[block, fill=iranblue!30, right=of lib] (trans) {\rl{گذار}};
    \node[block, fill=olgoogreen!50, right=of trans] (consol) {\rl{تثبیت دموکراسی}};
    
    \draw[arrow] (auth) -- (lib);
    \draw[arrow] (lib) -- (trans);
    \draw[arrow] (trans) -- (consol);
\end{tikzpicture}
\captionof{figure}{\rl{مراحل گذار دموکراتیک}}
\end{center}

\section{چارچوب ترکیبی پیشنهادی}

برای ایران، ترکیبی از نظریه‌ها مناسب‌تر است:

\begin{empirebox}
\begin{itemize}
    \item \textbf{از اسکاچپول:} توجه به بحران‌های ساختاری نظام
    \item \textbf{از تیلی:} تمرکز بر بسیج منابع و فرصت سیاسی
    \item \textbf{از شارپ:} تاکتیک‌های غیرخشونت‌آمیز
    \item \textbf{از گذارشناسان:} برنامه‌ریزی برای دوره انتقالی
\end{itemize}
\end{empirebox}
