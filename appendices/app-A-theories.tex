% app-A-theories.tex
% پیوست الف: نظریه‌های انقلاب و گذار

\chapter{نظریه‌های انقلاب و گذار}
\label{app:theories}

\begin{center}
    \includegraphics[width=0.6\textwidth]{images/diagrams/theories.png}
\end{center}

\section{نظریه ساختاری اسکاچپول}

تدا اسکاچپول در کتاب «دولت‌ها و انقلاب‌های اجتماعی» (۱۹۷۹) استدلال می‌کند که انقلاب‌ها زمانی رخ می‌دهند که:

\begin{itemize}
    \item دولت دچار بحران مالی و نظامی شود
    \item نخبگان از حکومت فاصله بگیرند
    \item دهقانان/توده‌ها ظرفیت سازماندهی داشته باشند
\end{itemize}

\begin{noktebox}
\textbf{کاربرد برای ایران:} بحران مالی و فشار بین‌المللی موجود است. شکاف نخبگان رو به افزایش. ظرفیت سازماندهی توده‌ها (به ویژه طبقه متوسط شهری) نسبتاً بالا.
\end{noktebox}

\section{نظریه بسیج منابع تیلی}

چارلز تیلی تأکید می‌کند که جنبش‌های اجتماعی نیازمند:
\begin{enumerate}
    \item منابع (پول، نیرو، سازمان)
    \item فرصت سیاسی
    \item چارچوب‌بندی مناسب (Framing)
\end{enumerate}

\section{۱۹۸ روش جین شارپ}

جین شارپ در «از دیکتاتوری به دموکراسی» ۱۹۸ روش مبارزه غیرخشونت‌آمیز را فهرست کرده:

\begin{table}[H]
\centering
\begin{tabular}{|R{3cm}|C{2cm}|L{6cm}|}
\hline
\rowcolor{olgoogreen!30}
\textbf{دسته} & \textbf{تعداد} & \textbf{نمونه} \\
\hline
اعتراض نمادین & ۵۴ & تظاهرات، امضا، نمادها \\
\hline
عدم همکاری اجتماعی & ۱۶ & تحریم، اعتصاب اجتماعی \\
\hline
عدم همکاری اقتصادی & ۴۹ & اعتصاب، تحریم خرید \\
\hline
عدم همکاری سیاسی & ۳۸ & تحریم انتخابات، نافرمانی \\
\hline
مداخله غیرخشونت‌آمیز & ۴۱ & تحصن، اشغال \\
\hline
\end{tabular}
\caption{دسته‌بندی روش‌های جین شارپ}
\end{table}

\section{انقلاب‌های مخملی: الزامات}

تجربه اروپای شرقی نشان می‌دهد انقلاب موفق غیرخشونت‌آمیز نیازمند:

\begin{olgoobox}
\begin{enumerate}
    \item رهبری متحد و شناخته‌شده
    \item سازمان‌های مدنی قوی
    \item رسانه‌های مستقل
    \item حمایت بین‌المللی
    \item شکاف در نیروهای امنیتی
    \item یک رویداد آغازگر
\end{enumerate}
\end{olgoobox}

\section{نظریه گذار دموکراتیک}

اودانل و اشمیتر چهار مرحله گذار را تعریف کردند:

\begin{center}
\begin{tikzpicture}[node distance=2.5cm]
    \node[block, fill=iranred!30] (auth) {اقتدارگرایی};
    \node[block, fill=irangold!30, right=of auth] (lib) {لیبرالیزاسیون};
    \node[block, fill=iranblue!30, right=of lib] (trans) {گذار};
    \node[block, fill=olgoogreen!50, right=of trans] (consol) {تثبیت دموکراسی};
    
    \draw[arrow] (auth) -- (lib);
    \draw[arrow] (lib) -- (trans);
    \draw[arrow] (trans) -- (consol);
\end{tikzpicture}
\captionof{figure}{مراحل گذار دموکراتیک}
\end{center}

\section{چارچوب ترکیبی پیشنهادی}

برای ایران، ترکیبی از نظریه‌ها مناسب‌تر است:

\begin{empirebox}
\begin{itemize}
    \item \textbf{از اسکاچپول:} توجه به بحران‌های ساختاری نظام
    \item \textbf{از تیلی:} تمرکز بر بسیج منابع و فرصت سیاسی
    \item \textbf{از شارپ:} تاکتیک‌های غیرخشونت‌آمیز
    \item \textbf{از گذارشناسان:} برنامه‌ریزی برای دوره انتقالی
\end{itemize}
\end{empirebox}
