% app-B-case-studies.tex
% پیوست ب: مطالعات موردی

\chapter{مطالعات موردی}
\label{app:case-studies}

\begin{center}
    \includegraphics[width=0.6\textwidth]{images/diagrams/case-studies.png}
\end{center}

\section{نمونه‌های موفق}

\subsection{لهستان: جنبش همبستگی (۱۹۸۰-۱۹۸۹)}

\begin{table}[H]
\centering
\begin{tabular}{|R{3cm}|L{9cm}|}
\hline
\rowcolor{olgoogreen!30}
\textbf{عامل} & \textbf{توضیح} \\
\hline
رهبری & لخ والسا - کارگر کاریزماتیک \\
\hline
سازمان & اتحادیه کارگری ۱۰ میلیونی \\
\hline
حمایت & کلیسای کاتولیک، غرب \\
\hline
تاکتیک & اعتصاب، مذاکره \\
\hline
نتیجه & انتخابات آزاد ۱۹۸۹ \\
\hline
\end{tabular}
\caption{خلاصه تجربه لهستان}
\end{table}

\textbf{درس برای ایران:} اهمیت سازمان‌دهی کارگری و حمایت نهاد مذهبی (که در ایران وجود ندارد).

\subsection{چکسلواکی: انقلاب مخملی (۱۹۸۹)}

\begin{itemize}
    \item مدت: تنها ۱۰ روز
    \item تاکتیک: تظاهرات مسالمت‌آمیز گسترده
    \item رهبری: واسلاو هاول - نمایشنامه‌نویس
    \item کلید: اعتصاب عمومی + ترک حزب کمونیست
\end{itemize}

\subsection{آفریقای جنوبی (۱۹۹۰-۱۹۹۴)}

\begin{olgoobox}[title=درس‌های کلیدی]
\begin{itemize}
    \item مذاکره ممکن است حتی با رژیم سرکوبگر
    \item عدالت انتقالی بدون انتقام (کمیسیون حقیقت)
    \item رهبری اخلاقی ماندلا
    \item فشار تحریم‌های بین‌المللی
\end{itemize}
\end{olgoobox}

\subsection{پرتغال: انقلاب میخک (۱۹۷۴)}

کودتای نظامی چپ‌گرا که به دموکراسی منجر شد. نشان‌دهنده امکان گذار از طریق نظامیان.

\section{نمونه‌های ناموفق}

\subsection{سوریه (۲۰۱۱-حال)}

\begin{enghelabbox}[title=چرا شکست خورد؟]
\begin{itemize}
    \item تفرقه شدید اپوزیسیون
    \item مسلح شدن زودهنگام
    \item مداخله روسیه و ایران
    \item فقدان حمایت غربی مؤثر
    \item شکاف‌های فرقه‌ای و قومی
\end{itemize}
\end{enghelabbox}

\textbf{درس:} اجتناب از مسلح شدن + وحدت اپوزیسیون

\subsection{ونزوئلا}

علی‌رغم فشار بین‌المللی و اپوزیسیون قوی، نظام مادورو دوام آورد:
\begin{itemize}
    \item حمایت روسیه و چین
    \item کنترل نظامیان بر اقتصاد
    \item سرکوب مؤثر
\end{itemize}

\section{جدول مقایسه‌ای}

\begin{table}[H]
\centering
\resizebox{\textwidth}{!}{
\begin{tabular}{|R{2.5cm}|C{1.5cm}|C{2cm}|C{2cm}|C{2cm}|C{2cm}|}
\hline
\rowcolor{iranblue!20}
\textbf{کشور} & \textbf{مدت} & \textbf{روش} & \textbf{حمایت خارجی} & \textbf{وحدت} & \textbf{نتیجه} \\
\hline
لهستان & ۹ سال & غیرخشن & غرب & بالا & موفق \\
\hline
چک & ۱۰ روز & غیرخشن & غرب & بالا & موفق \\
\hline
آفریقای جنوبی & ۴ سال & مذاکره & غرب & بالا & موفق \\
\hline
سوریه & ۱۳+ سال & مسلح & محدود & پایین & شکست \\
\hline
ونزوئلا & ۶+ سال & مختلط & غرب & متوسط & جاری \\
\hline
\textbf{ایران} & - & \textbf{؟} & \textbf{بالقوه} & \textbf{پایین} & \textbf{؟} \\
\hline
\end{tabular}
}
\caption{مقایسه نمونه‌ها با ایران}
\end{table}

\section{نتیجه‌گیری}

\begin{empirebox}
عوامل کلیدی موفقیت:
\begin{enumerate}
    \item وحدت اپوزیسیون
    \item خشونت‌پرهیزی
    \item حمایت بین‌المللی
    \item شکاف در نظام
    \item آمادگی برای دوره انتقالی
\end{enumerate}
\end{empirebox}
