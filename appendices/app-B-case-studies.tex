% app-B-case-studies.tex
% پیوست ب: مطالعات موردی

\chapter{\rl{مطالعات موردی}}
\label{app:case-studies}

\begin{center}
\begin{tikzpicture}[scale=0.9, transform shape]
    \node[concept] (base) at (0,0) {\rl{مدل‌های گذار}};
    \node[step, fill=irangreen!20] (m1) at (-4, 2) {\rl{لهستان: همبستگی}};
    \node[step, fill=iranblue!20] (m2) at (4, 2) {\rl{چک: انقلاب مخملی}};
    \node[step, fill=iranred!20] (m3) at (0, -3) {\rl{سوریه: جنگ داخلی}};
    
    \draw[connection] (base) -- (m1);
    \draw[connection] (base) -- (m2);
    \draw[connection] (base) -- (m3);
\end{tikzpicture}
\captionof{figure}{\rl{توزیع الگوهای موفق و ناموفق گذار در جهان}}
\end{center}

\section{\rl{نمونه‌های موفق}}

\subsection{\rl{لهستان: جنبش همبستگی (۱۹۸۰-۱۹۸۹)}}

\begin{table}[H]
\centering
\begin{tabular}{|R{4cm}|R{11cm}|}
\hline
\rowcolor{olgoogreen!30}
\textbf{\rl{عامل}} & \textbf{\rl{توضیح}} \\
\hline
\rl{رهبری} & \rl{لخ والسا - کارگر کاریزماتیک} \\
\hline
\rl{سازمان} & \rl{اتحادیه کارگری ۱۰ میلیونی} \\
\hline
\rl{حمایت} & \rl{کلیسای کاتولیک، غرب} \\
\hline
\rl{تاکتیک} & \rl{اعتصاب، مذاکره} \\
\hline
\rl{نتیجه} & \rl{انتخابات آزاد ۱۹۸۹} \\
\hline
\end{tabular}
\caption{\rl{خلاصه تجربه لهستان}}
\end{table}

\textbf{\rl{درس برای ایران:}} \rl{اهمیت سازمان‌دهی کارگری و حمایت نهاد مذهبی (که در ایران وجود ندارد).}

\subsection{\rl{چکسلواکی: انقلاب مخملی (۱۹۸۹)}}

\begin{itemize}
    \item \rl{مدت: تنها ۱۰ روز}
    \item \rl{تاکتیک: تظاهرات مسالمت‌آمیز گسترده}
    \item \rl{رهبری: واسلاو هاول - نمایشنامه‌نویس}
    \item \rl{کلید: اعتصاب عمومی + ترک حزب کمونیست}
\end{itemize}

\subsection{\rl{آفریقای جنوبی (۱۹۹۰-۱۹۹۴)}}

\begin{olgoobox}[title=\rl{درس‌های کلیدی}]
\begin{itemize}
    \item \rl{مذاکره ممکن است حتی با رژیم سرکوبگر}
    \item \rl{عدالت انتقالی بدون انتقام (کمیسیون حقیقت)}
    \item \rl{رهبری اخلاقی ماندلا}
    \item \rl{فشار تحریم‌های بین‌المللی}
\end{itemize}
\end{olgoobox}

\subsection{\rl{پرتغال: انقلاب میخک (۱۹۷۴)}}

\rl{کودتای نظامی چپ‌گرا که به دموکراسی منجر شد. نشان‌دهنده امکان گذار از طریق نظامیان.}

\section{\rl{نمونه‌های ناموفق}}

\subsection{\rl{سوریه (۲۰۱۱-حال)}}

\begin{enghelabbox}[title=\rl{چرا شکست خورد؟}]
\begin{itemize}
    \item \rl{تفرقه شدید اپوزیسیون}
    \item \rl{مسلح شدن زودهنگام}
    \item \rl{مداخله روسیه و ایران}
    \item \rl{فقدان حمایت غربی مؤثر}
    \item \rl{شکاف‌های فرقه‌ای و قومی}
\end{itemize}
\end{enghelabbox}

\textbf{\rl{درس:}} \rl{اجتناب از مسلح شدن + وحدت اپوزیسیون}

\subsection{\rl{ونزوئلا}}

\rl{علی‌رغم فشار بین‌المللی و اپوزیسیون قوی، نظام مادورو دوام آورد:}
\begin{itemize}
    \item \rl{حمایت روسیه و چین}
    \item \rl{کنترل نظامیان بر اقتصاد}
    \item \rl{سرکوب مؤثر}
\end{itemize}

\section{\rl{جدول مقایسه‌ای}}

\begin{table}[H]
\centering
\resizebox{\textwidth}{!}{
\begin{tabular}{|R{3.5cm}|C{2.5cm}|C{2.5cm}|C{3cm}|C{2.5cm}|C{2.5cm}|}
\hline
\rowcolor{iranblue!20}
\textbf{\rl{کشور}} & \textbf{\rl{مدت}} & \textbf{\rl{روش}} & \textbf{\rl{حمایت خارجی}} & \textbf{\rl{وحدت}} & \textbf{\rl{نتیجه}} \\
\hline
\rl{لهستان} & \rl{۹ سال} & \rl{غیرخشن} & \rl{غرب} & \rl{بالا} & \rl{موفق} \\
\hline
\rl{چک} & \rl{۱۰ روز} & \rl{غیرخشن} & \rl{غرب} & \rl{بالا} & \rl{موفق} \\
\hline
\rl{آفریقای جنوبی} & \rl{۴ سال} & \rl{مذاکره} & \rl{غرب} & \rl{بالا} & \rl{موفق} \\
\hline
\rl{سوریه} & \rl{۱۳+ سال} & \rl{مسلح} & \rl{محدود} & \rl{پایین} & \rl{شکست} \\
\hline
\rl{ونزوئلا} & \rl{۶+ سال} & \rl{مختلط} & \rl{غرب} & \rl{متوسط} & \rl{جاری} \\
\hline
\textbf{\rl{ایران}} & - & \textbf{؟} & \textbf{\rl{بالقوه}} & \textbf{\rl{پایین}} & \textbf{؟} \\
\hline
\end{tabular}
}
\caption{\rl{مقایسه نمونه‌ها با ایران}}
\end{table}

\section{\rl{نتیجه‌گیری}}

\begin{empirebox}
\rl{عوامل کلیدی موفقیت:}
\begin{enumerate}
    \item وحدت اپوزیسیون
    \item خشونت‌پرهیزی
    \item حمایت بین‌المللی
    \item شکاف در نظام
    \item آمادگی برای دوره انتقالی
\end{enumerate}
\end{empirebox}
